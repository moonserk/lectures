\documentclass[a4paper]{article}
\usepackage[T1,T2A]{fontenc}
\usepackage[utf8]{inputenc}
\usepackage[english,russian]{babel}
\usepackage{booktabs}
\usepackage{color,colortbl}
%\usepackage{amsmath}
%\usepackage{amsfonts}
%\usepackage{amssymb}
%\usepackage{makeidx}

\definecolor{darkishgreen}{RGB}{39,203,22}
\definecolor{LightCyan}{rgb}{0.88,1,1}
\definecolor{Gray}{gray}{0.9}
\definecolor{lightRed}{RGB}{230,170,150}
\definecolor{modRed}{RGB}{230,82,90}
\definecolor{strongRed}{RGB}{230,6,6}

\usepackage[english,russian]{babel}

\begin{document}

\section{Информатика, ее составляющие и роль в современном мире}

\subsection{Информатика, ее место в системе наук}

Термин информатика возник в 60-х гг. во Франции для названия области, занимающейся автоматизированной обработкой информации с помощью электронных вычислительных машин. Французский термин informatigue (информатика)образован путем слияния слов information (информация) и automatigue (автоматика) и означает 'информационная автоматика или автоматизированная переработка информации'. В англоязычных странах этому термину соответствует синоним computer science (наука о компьютерной технике).

Выделение информатики как самостоятельной области человеческой деятельности в первую очередь связано с развитием компьютерной техники. Причем основная заслуга в этом принадлежит микропроцессорной технике, появление которой в середине 70-х гг. послужило началом второй электронной революции. С этого времени элементной базой вычислительной машины становятся интегральные схемы и микропроцессоры, а область, связанная с созданием и использованием компьютеров, получила мощный импульс в своем развитии. Термин 'информатика' приобретает новое дыхание и используется не только для отображения достижений компьютерной техники, но и связывается с процессами передачи и обработки информации.

В нашей стране подобная трактовка термина 'информатика' утвердилась с момента принятия решения в 1983 г. на сессии годичного собрания Академии наук СССР об организации нового отделения информатики, вычислительной техники и автоматизации. Информатика трактовалась как 'комплексная научная и инженерная дисциплина, изучающая все аспекты разработки, проектирования, создания, оценки, функционирования основанных на ЭВМ систем переработки информации, их применения и воздействия на различные области социальной практики'. Информатика в таком понимании нацелена на разработку общих методологических принципов построения информационных моделей. Поэтому методы информатики применимы всюду, где существует возможность описания объекта, явления, процесса и т.п. с помощью информационных моделей.

Существует множество определений информатики, что связано с многогранностью ее функций, возможностей, средств и методов. Один из вариантов которые предлагает википедия: Информáтика (фр. Informatique) — наука о методах и процессах сбора, хранения, обработки, передачи, анализа и оценки информации с нием компьютерных технологий, обеспечивающих возможность её использования для принятия решений. Информатика включает дисциплины, относящиеся к обработке информации в вычислительных машинах и вычислительных сетях: как абстрактные, вроде анализа алгоритмов, так и конкретные, например, разработка языков программирования и протоколов передачи данных.

Темами исследований в информатике как науки являются вопросы: что можно, а что нельзя реализовать в программах и базах данных (теория вычислимости и искусственный интеллект), каким образом можно решать специфические вычислительные и информационные задачи с максимальной эффективностью (теория сложности вычислений), в каком виде следует хранить и восстанавливать информацию специфического вида (структуры и базы данных), как программы и люди должны взаимодействовать друг с другом (пользовательский интерфейс и языки программирования и представление знаний) и т. п.

Информатика обычно представляют состоящей из двух частей:

Теоретическая информатика

Огромное поле исследований теоретической информатики включает как классическую теорию алгоритмов, так и широкий спектр тем, связанных с более абстрактными логическими и математическими аспектами вычислений. Теоретическая информатика занимается теориями формальных языков, автоматов, алгоритмов, вычислимости и вычислительной сложности, а также вычислительной теорией графов, криптологией, логикой (включая логику высказываний и логику предикатов), формальной семантикой и закладывает теоретические основы для разработки компиляторов языков программирования.

Прикладная информатика

Прикладная информатика направлена на применение понятий и результатов теоретической информатики к решению конкретных задач в конкретных прикладных областях.

\subsection{Связь информатики с другими науками}

Информатика — очень широкая сфера научных знаний, возникшая на стыке нескольких фундаментальных и прикладных дисциплин. Фундаментальная наука — наука, изучающая объективные законы природы и общества, осуществляющая теоретическую систематизацию знаний о действительности. К фундаментальным принято относить те науки, основные понятия которых носят общенаучный характер, используются во многих других науках и видах деятельности. Как комплексная научная дисциплина информатика связана с:

\begin{itemize}
\item философией и психологией — через учение об информации и теорию познания;
\item математикой — через теорию математического моделирования, дискретную математику, математическую логику и теорию алгоритмов;
\item лингвистикой — через учение о формальных языках и о знаковых системах;
\item кибернетикой — через теорию информации и теорию управления;
\item физикой и химией, электроникой и радиотехникой — через «материальную» часть компьютера и информационных систем.
\end{itemize}
Роль информатики в развитии общества чрезвычайно велика. Она является научным фундаментом процесса информатизации общества. С ней связано прогрессивное увеличение возможностей компьютерной техники, развитие информационных сетей, создание новых информационных технологий, которые приводят к значительным изменениям во всех сферах общества: в производстве, науке, образовании, медицине и т.д.

\subsection{Экономические, социальные и правовые аспекты информационных технологий}

Роль информатики в развитии общества чрезвычайно велика. С ней связано начало революции в области накопления, передачи и обработки информации. Эта революция, следующая за революциями в овладении веществом и энергией, затрагивает и коренным образом преобразует не только сферу материального производства, но и интеллектуальную, духовную сферы жизни. Рост производства компьютерной техники, развитие информационных сетей, создание новых информационных технологий приводят к значительным изменениям во всех сферах общества: в производстве, науке, образовании, медицине и т.д.

Информатизация общества — организованный социально-экономический и научно-технический процесс создания оптимальных условий для удовлетворения информационных потребностей и реализации прав граждан, органов государственной власти, органов местного самоуправления организаций, общественных объединений на основе формирования и использования информационных ресурсов.
Цель информатизации — улучшение качества жизни людей за счет увеличения производительности и облегчения условий их труда.
Информатизация — это сложный социальный процесс, связанный со значительными изменениями в образе жизни населения. Он требует серьёзных усилий на многих направлениях, включая ликвидацию компьютерной неграмотности, формирование культуры использования новых информационных технологий и др.
Бурное развитие информационных технологий имеет и негативные аспекты:
Расслоение общества на 'продвинутых специалистов' и тех, кто не может угнаться за быстрыми изменениями условий жизни (требуется изучать все новые средства связи, программные продукты).
Вторжение в частную жизнь людей и организаций, шпионаж с помощью технических средств.

\subsubsection{Правовая охрана программ и данных.}

Правовая охрана программ для ЭВМ и баз данных впервые в полном объеме введена в Российской Федерации Законом «О правовой охране программ для электронных вычислительных машин и баз данных», который вступил в силу 20 октября 1992 г. Предоставляемая настоящим законом правовая охрана распространяется на все виды программ для компьютеров (в том числе на операционные системы и программные комплексы), которые могут быть выражены на любом языке и в любой форме. Для признания и реализации авторского права на компьютерную программу не требуется ее регистрация в какой-либо организации. Авторское право на компьютерную программу возникает автоматически при ее создании. Для оповещения о своих правах разработчик программы может, начиная с первого выпуска в свет программы, использовать знак охраны авторского права, состоящий из трех элементов:

\begin{itemize}
\item буквы С в окружности или круглых скобках \textcopyright;
\item наименования (имени) правообладателя;
\item года первого выпуска программы.       
\end{itemize}
Автору программы принадлежит исключительное право на воспроизведение и распространение программы любыми способами, а также на осуществление модификации программы.y

\end{document}
