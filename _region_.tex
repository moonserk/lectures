\message{ !name(CS_labs.tex)}
\message{ !name(CS_labs.tex) !offset(-2) }
%\documentclass[a4paper]{article}
%\usepackage[T1,T2A]{fontenc}
%\usepackage[utf8]{inputenc}
%\usepackage[english,russian]{babel}
%\usepackage{booktabs}
%\usepackage{color,colortbl}
%\usepackage{amsmath}
%\usepackage{amsfonts}
%\usepackage{amssymb}
%\usepackage{makeidx}

%\definecolor{darkishgreen}{RGB}{39,203,22}
%\definecolor{LightCyan}{rgb}{0.88,1,1}
%\definecolor{Gray}{gray}{0.9}
%\definecolor{lightRed}{RGB}{230,170,150}
%\definecolor{modRed}{RGB}{230,82,90}
%\definecolor{strongRed}{RGB}{230,6,6}

%\usepackage[english,russian]{babel}

%\begin{document}

\section{Лабораторная работа №1. Операции с двоичными числами}

Цель работы:

\subsection{Теоретические основы}

В основу работы ЭВМ положены арифметические и логические операции с двоичными числами. Правила выполнения арифметических действий над двоичными числами можно представить таблицами сложения, вычитания и умножения. Все действия в двоичной арифметике сводятся к поразрядному выполнению трёх указанных операций в таблице~\ref{tab:mytab}.

Правила арифметики во всех позиционных системах счисления аналогичны. В двоичной системе арифметическое сложение происходит так же, как в десятичной системе с учетом переноса единицы в старший разряд.

\begin{table}[h]
  \caption{Арифметические действия над двоичными числамин}
  \begin{center}\label{tab:mytab}
    \begin{tabular}{|c|c|c|}
      \hline
     Сложение & Вычитание & Умножение \\
     \hline
     $0 + 0 = 0$ & $0 - 0 = 0$ & $0 * 0 = 0$\\
     $0 + 1 = 1$ & $0 - 1 = -11$ & $0 * 1 = 0$\\
     $1 + 0 = 1$ & $1 - 1 = 1$ & $1 * 0 = 0$\\
      $1 + 1 = 10$ & $1 - 1 = 0$ & $1 * 1 = 1$\\
      \hline
     \end{tabular}
  \end{center}       
\end{table}

\begin{center}

Пример 1. Выполнить операцию арифметического сложения.
\[
\begin{array}{r}
+
\begin{array}{r}
01101\\
00111\\
\end{array} \\
\midrule
\begin{array}{r}
10100
\end{array}
\end{array}
\]
Пример 2. Выполнить операцию арифметического вычитания.
\[
\begin{array}{r}
-
\begin{array}{r}
01101\\
00111\\
\end{array} \\
\midrule
\begin{array}{r}
00110
\end{array}
\end{array}
\]
Пример 3. Выполнить операцию арифметического умножения.

\[
\begin{array}{r}
*
\begin{array}{r}
0001101\\
0000111\\
\end{array} \\
\midrule
\begin{array}{r}
1011011
\end{array}
\end{array}
\]

\end{center}

Следует заметить, что в реальных ЭВМ чаще всего используются 16-, 32-, 64-разрядные сетки (машинные слова). Однако для учебных целей рассматривается простой вариант выполнения операции сложения.


Очень часто в вычислениях должны использоваться не только положительные, но и отрицательные числа. Число со знаком в вычислительной технике представляется путем представления старшего разряда числа в качестве знакового. Принято считать, что 0 в знаковом разряде означает знак «плюс» для данного числа, а 1 – знак «минус». Выполнение арифметических операций над числами с разными знаками представляется для аппаратной части довольно сложной процедурой. В этом случае нужно определить большее по модулю число, произвести вычитание и присвоить разности знак большего по модулю числа. Применение дополнительного кода позволяет выполнить операцию алгебраического суммирования и вычитания на обычном сумматоре. При этом не требуется определения модуля и знака числа. \linebreak
\textbf{Прямой код} пyредставляет собой одинаковое представление значимой части числа для положительных и отрицательных чисел и отличается только знаковым битом. В прямом коде число 0 имеет два представления «+0» и «–0». \linebreak
\textbf{Обратный код} для положительных чисел имеет тот же вид, что и прямой код, а для отрицательных чисел образуется из прямого кода положительного числа путем инвертирования всех значащих разрядов прямого кода. В обратном коде число 0 также имеет два представления «+0» и «–0». \linebreak
\textbf{Дополнительный код} для положительных чисел имеет тот же вид, что и прямой код, а для отрицательных чисел образуется путем прибавления 1 к обратному коду. Добавление 1 к обратному коду числа 0 дает единое представление числа 0 в дополнительном коде. Однако это приводит к асимметрии диапазонов представления чисел относительно нуля. Так, в восьмиразрядном представлении диапазон изменения чисел с учетом знака. \linebreak

\[-128 <= x <= 127\]

\subsubsection{Сложение и вычитание чисел со знаком в дополнительном коде}

В ЭВМ вычитание заменяется сложением чисел в обратном или в дополнительном коде, что позволяет и для вычитания и для сложения использовать одно и тожк устройство --- сумматор.

При вычитании чисел при помощи обратного кода, после операции сложения в обратном коде получается промежуточный результат, к которому нужно прибавить значение разряда переноса, после чего отбросить этот разряд.

В дополнительном коде в отличие от обратного возможный перенос учитывается в самих числах, а не в результате. После операциисложения в дополнительном коде получается промежуточный результат, от которого нужно отбросить разряд переноса.

\newpage
\begin{table}[h]
      \caption{Пример сложение чисел в двоичной системе счисления}
      \begin{center}\label{tab:add}
      \begin{tabular}{|c| * {11}{c|}}
        \hline
        Номер бита & Перенос 8 & 7 & 6 & 5 & 4 & 3 & 2 & 1 & 0 \\
        \hline
        Число А ($43 = 32 + 8 + 2 + 1$) &  & 0 & 0 & 1 & 0 & 1 & 0 & 1 & 1\\
        \hline
        Число B ($29 = 16 + 8 + 4 + 1$) &  & 0 & 0 & 0 & 1 & 1 & 1 & 0 & 1\\
        \hline
        Перенос из разряда &  &  & 1 & 1 & 1 & 1 & 1 & 1 & \\
        \hline
        Сумма ($72 = 43 + 29$) &  & 0 & 1 & 0 & 0 & 1 & 0 & 0 & 0 \\
        \hline
      \end{tabular}
    \end{center}
\end{table}


\begin{table}[h]
      \caption{Пример вычитания в двоичной системе счисления с использованрием обратного кода}
      \begin{center}\label{tab:sub_OB}
        %\toprule
      \begin{tabular}{|c| * {11}{c|}}
        \hline
        Номер бита & Перенос 8 & 7 & 6 & 5 & 4 & 3 & 2 & 1 & 0 \\
        \hline
        Модуль числа 29 ($29 = 16 + 8 + 4 + 1$) &  & 0 & 0 & 0 & 1 & 1 & 1 & 0 & 1\\
        \hline
        Обратый код числа 29 ($29 = 16 + 8 + 4 + 1$) &  & 1 & 1 & 1 & 0 & 0 & 0 & 1 & 0\\
        \hline
        Число А ($43 = 32 + 8 + 2 + 1$) &  & 0 & 0 & 1 & 0 & 1 & 0 & 1 & 1\\
        \hline
        Перенос из разряда &  &  & 1 &  &  &  & 1 &  & \\
        \hline
        Сумма (промежуточный результат) & 1 & 0 & 0 & 0 & 0 & 1 & 1 & 0 & 1 \\
        \hline
        + перенос &  &  &  &  &  &  &  &  & 1 \\
        \hline
        Сумма (окончательный результат) &  & 0 & 0 & 0 & 0 & 1 & 1 & 1 & 0 \\
        \hline
      \end{tabular}
    \end{center}
\end{table}

\begin{table}[h]
      \caption{Пример вычитания в двоичной системе счисления с использованрием дополнительного кода}
      \begin{center}\label{tab:sub_DP}
        %\toprule
      \begin{tabular}{|c * {11}{c|}}
        \hline
        Номер бита & Перенос 8 & 7 & 6 & 5 & 4 & 3 & 2 & 1 & 0 \\
        \hline
        Модуль числа 29 ($29 = 16 + 8 + 4 + 1$) &  & 0 & 0 & 0 & 1 & 1 & 1 & 0 & 1\\
        \hline
        Обратый код числа 29 ($29 = 16 + 8 + 4 + 1$) &  & 1 & 1 & 1 & 0 & 0 & 0 & 1 & 0\\
        \hline
        +1 ($29 = 16 + 8 + 4 + 1$) &  &  &  &  &  &  &  &  & 1\\
        \hline
        Дополнительный код числа 29 ($29 = 16 + 8 + 4 + 1$) &  & 1 & 1 & 1 & 0 & 0 & 0 & 1 & 1\\
        \hline
        Число А ($43 = 32 + 8 + 2 + 1$) &  & 0 & 0 & 1 & 0 & 1 & 0 & 1 & 1\\
        \hline
        Перенос из разряда &  & 1 & 1 &  &  &  & 1 & 1 & \\
        \hline
        Сумма (промежуточный результат) & 1 & 0 & 0 & 0 & 0 & 1 & 1 & 1 & 0 \\
        \hline
        + перенос &  &  &  &  &  &  &  &  & 1 \\
        \hline
        Сумма (окончательный результат) &  & 0 & 0 & 0 & 0 & 1 & 1 & 1 & 0 \\
        \hline
      \end{tabular}
    \end{center}
\end{table}

\newpage
\subsection{Задание на лабораторную работу}

Для выполнения работы необходимо:
\begin{enumerate}
  \item Изучить теоретический материал
  \item Из таблицы~\ref{tab:tasks} выбрать числовые данные в соответствии с вашим вариантом.
  \item Перевести десячичные числа в двоичные
  \item Выполнить указанные операции над ними ()
  \item Записать решения по примеру таблиц~\ref{tab:add},~\ref{tab:sub_OB},~\ref{tab:sub_DP}
  \item Оформить отчет по лабораторной работе
\end{enumerate}

\begin{table}
      \caption{Варианты заданий}
      \begin{center}\label{tab:tasks}
        %\toprule
      \begin{tabular}{|c|cccccc|}
        \hline
        Вариант & 1 & 2 & 3 & 4 & 5 & 6\\
        \hline
        1 & 2+2 & 34+32 & 6-4 & 64-32 & 89-23 & 10-40\\
        \hline
        2 & 2+3 & 12+64 & 6-2 & 23-12 & 128-89 & 15-31\\
        \hline
        3 & 4+7 & 14+65 & 4-2 & 54-23 & 100-78 & 16-32\\
        \hline
        4 & 3+5 & 15+32 & 4-1 & 56-44 & 90-32 & 2-19\\
        \hline
        5 & 5+3 & 44+12 & 10-3 & 78-34 & 123-120 & 15-16\\
        \hline
        6 & 6+9 & 8+21 & 3-1 & 79-45 & 111-11 & 31-32\\
        \hline
        7 & 6+4 & 12+21 & 4-1 & 79-10 & 111-100 & 54-63\\
        \hline
        8 & 7+2 & 56+22 & 5-2 & 39-14 & 128-28 & 22-64\\
        \hline
        9 & 9+3 & 23+32 & 6-5 & 55-22 & 120-30 & 23-88\\
        \hline
        10 & 4+7 & 45+32 & 9-3 & 88-22 & 90-8 & 16-31\\
        \hline
        11 & 3+4 & 32+32 & 8-2 & 44-21 & 123-33 & 32-63\\
        \hline
        12 & 3+2 & 64+16 & 10-5 & 78-67 & 129-88 & 15-38\\
        \hline
        13 & 4+5 & 16+32 & 10-2 & 81-64 & 148-64 & 44-88\\
        \hline
        14 & 7+7 & 23+24 & 2-1 & 64-16 & 63-15 & 57-78\\
        \hline
        15 & 5+5 & 54+40 & 7-5 & 55-12 & 120-15 & 76-100\\
        \hline
        16 & 6+6 & 41+23 & 7-3 & 65-15 & 115-31 & 55-105\\
        \hline
      \end{tabular}
    \end{center}
\end{table}
                                     
\subsection{Содержание отчета}
\begin{enumerate}
  \item Титульный Лист
  \item Цель работы
  \item Краткие теоретические сведения по теме лабораторной работы
  \item выполненное задание
  \item краткий вывод о проделанной работе
\end{enumerate}

\subsection{Литература}


%\end{document}

\message{ !name(CS_labs.tex) !offset(-265) }
