\documentclass[a4paper]{article}
\usepackage[T1,T2A]{fontenc}
\usepackage[utf8]{inputenc}
\usepackage[english,russian]{babel}
\usepackage{booktabs}
\usepackage{color,colortbl}
%\usepackage{amsmath}
%\usepackage{amsfonts}
%\usepackage{amssymb}
%\usepackage{makeidx}

\definecolor{darkishgreen}{RGB}{39,203,22}
\definecolor{LightCyan}{rgb}{0.88,1,1}
\definecolor{Gray}{gray}{0.9}
\definecolor{lightRed}{RGB}{230,170,150}
\definecolor{modRed}{RGB}{230,82,90}
\definecolor{strongRed}{RGB}{230,6,6}

\usepackage[english,russian]{babel}

\begin{document}

\section{Информационные процессы и технологии.}

\subsection{Введение}

Прежде всего определим что такое вычислительная машна (ЭВМ). Интуитивно понятно, что это средство для автоматизации вычислений. Однако вычислительные машины используются настолько широко что по неволе возникает сомнение в правильности интуитивного определения.

В ``Толковом словаре по информатике'' приведено следующее определение: ЭВМ --- это комплекс технических, аппаратных и программных средств, предназначенных для автоматической обработки информации, вычислений, автоматического управления. Таким образом, понятие ЭВМ тесно связано с понятиями ``информация'', ``вычисления'', ``алгоритмическая обработка''.

Объект передачи и преобразования в вычислительных системах --- \textit{информация}. Все процессы, происходящие в вычислительной системе, связаны непосредственно с различными физическими носителями информации, и все узлы и блоки этой системы явлются физической средой, в которой осуществляются информационные процессы. Информационные процессы состоят не только в передаче, но и в переобразовании, переработке и хранении информации. Все это состовляет предмет науки информатики. \textbf{Информатика} состоит из трех составных частей:
\begin{itemize}
  \item теории передачи и преобразования информации;
  \item алгоритмических средств обработки информации;
  \item вычислительных средств;
\end{itemize}

\subsection{Информационные процессы}

В ходе информационного процесса данные преобразуются из одного вида в другой с помощью методов. Обработка данных включает в себя множество различных операций. В структуре возможных операций с данными можно выделить основыне:

\begin{itemize}
\item сбор данных – накопление данных с целью обеспечения достаточной полноты информации для принятия решения;

\item формализация данных – приведение данных, поступающих из разных источников, к одинаковой форме, чтобы сделать их сопоставимыми между собой, то есть повысить их уровень доступности;

\item фильтрация данных – отсеивание «лишних» данных, в которых нет необходимости для принятия решений; при этом должен уменьшаться уровень «шума», а достоверность и адекватность данных должны возрастать;

\item сортировка данных – упорядочение данных по заданному признаку с целью удобства использования; повышает доступность информации;

\item группировка данных – объединение данных по заданному признаку с целью повышения удобства использования; повышает доступность информации;

\item архивация данных – организация хранения данных в удобной и легкодоступной форме; служит для снижения экономических затрат на хранение данных и повышает общую надежность информационного процесса в целом;

\item защита данных – комплекс мер, направленных на предотвращение утраты, воспроизведение и модификации данных;

\item транспортировка данных – прием и передача (доставка и поставка) данных между удаленными участниками информационного процесса; при этом источник данных в информатике принято называть сервером, а потребителя – клиентом;

\item преобразование данных – перевод данных из одной формы в другую или из одной структуры в другую. Пример: изменение типа носителя; книги – бумага, электронная форма, микрофотоплёнка. Необходимость в многократном преобразовании данных возникает также при их транспортировке, особенно если она осуществляется средствами, не предназначенными для транспортировки данного вида данных.

\end{itemize}


\subsubsection{Позиционные системы счисления.}



\subsubsection{Цифровое представление данных}

В вычислительной технике используется двоичная система кодирования, основанная на предствалении данных последовательностью всего двух знкаов: 0 и 1. Эти знаки называются двоичными цифрами (binary digit) --- или битом (bit). Особенность этой системы то что она является \textit{простейшей} из всех возможных систем счисления, если мы захотим еще упростить нам придется оставить только 0, а он сам по себе не представляет никакой информации.

\subsubsection{Единицы измерения информации. Разновидности носителей информации.}





\subsection{Архитектура ЭВМ}

\subsection{Информационные системы}

\end{document}
