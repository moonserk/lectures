\documentclass[a4paper]{article}
\usepackage[T1,T2A]{fontenc}
\usepackage[utf8]{inputenc}
\usepackage[english,russian]{babel}
\usepackage{booktabs}
\usepackage{color,colortbl}
%\usepackage{amsmath}
%\usepackage{amsfonts}
%\usepackage{amssymb}
%\usepackage{makeidx}
% далее идёт преамбула
\usepackage{tikz}
\usetikzlibrary{graphs}

\definecolor{darkishgreen}{RGB}{39,203,22}
\definecolor{LightCyan}{rgb}{0.88,1,1}
\definecolor{Gray}{gray}{0.9}
\definecolor{lightRed}{RGB}{230,170,150}
\definecolor{modRed}{RGB}{230,82,90}
\definecolor{strongRed}{RGB}{230,6,6}

\usepackage[english,russian]{babel}

\begin{document}

\section{Офисные средства  обработки информации}

\subsection{Текстовые процессоры}
Текстовый процессор – компьютерная программа, предназначенная для создания и редактирования текстовых документов, компоновки макета текста и предварительного просмотра документов в том виде, в котором они будут напечатаны (свойство, известное как WYSIWYG).

Современные текстовые процессоры позволяют выполнять форматирование шрифтов и абзацев, проверку орфографии, создание и вставка таблиц и графических объектов, а также включают некоторые возможности настольных издательских систем. Текстовые процессоры используют в случаях, когда кроме содержания текста имеет значение и его внешний вид (подготовка официальных документов). Документ, созданный с помощью текстового процессора, содержит кроме текста еще и информацию о его форматировании, которая сохраняется в кодах, не видимых пользователю. Поскольку разными текстовыми процессорами используются для оформления текста разные коды (документы с разными форматами), то перенос форматированных текстовых документов из одного текстового процессора в другой не всегда является корректным. В таких случаях форматирование может быть сохранено лишь частично (как, например, при переносе документа из MS Word в OpenOffice Writer) или вообще не сохраниться (переносится только текст). Тогда необходимо форматирование документа выполнять заново.

Популярные текстовые процессоры Microsoft Word – мощный текстовый процессор, предназначенный для создания, просмотра и редактирования текстовых документов. Программа входит в пакет Microsoft Office. Выпускается с 1983 г. Текущая версия MS Word 2016 для Windows и MS Word 2011 для Mac. Возможности программы Word расширены встроенным макроязыком Visual Basic (VBA). Однако это предоставляет дополнительные возможности для написания встраиваемых в документы вирусов, которые называются макровирусами.

WordPad – текстовый процессор, который входит в состав операционной системы Microsoft Windows. Гораздо мощнее программы Блокнот, но уступает полноценному текстовому процессору Microsoft Word. Процессор поддерживает форматирование и печать текста, но не имеет инструментов для создания таблиц, средств проверки орфографии.

LaTeX – наиболее популярный макропакет системы компьютерной вёрстки TeX для облегчения набора сложных документов. Пакет предназначен для автоматизации многих задач набора текста (на нескольких языках) и подготовки статей, нумерации разделов и формул, перекрёстных ссылок, размещения иллюстраций и таблиц на странице, ведения библиографии и др.

OpenOffice.org Writer – текстовый процессор, который входит в состав пакета свободного программного обеспечения OpenOffice.org. Writer во многом аналогичен текстовому процессору Microsoft Word, но имеет некоторые возможности, которые отсутствуют в Word (например, поддержка стилей страниц).

\subsection{Табличные процессоры}
Табличный процессор - это программа для обработки информации, которую можно представить в виде таблиц. Табличные процессоры позволяют не только создавать на компьютере таблицы, но и проводить автоматизацию обработки данных, внесенных в таблицы.

Это позволяет повысить эффективность работы и осуществлять ее на более высоком качественном уровне.

С помощью табличных процессоров можно делать расчеты по экономике, бухгалтерскому делу, а также в различных областях инженерного дела.

Также табличные процессоры позволяют строить диаграммы и графики, с помощью них можно проводить экономический анализ, создавать модели различных ситуаций с количественной точки зрения и многое другое. Хранение и обработка информации в табличных процессорах осуществляется в виде двумерных массивов, которые состоят из строк и столбцов. Такие массивы называются рабочими листами, которые входят в рабочую книгу.


\end{document}
