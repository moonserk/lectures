\documentclass[a4paper]{article}
\usepackage[T1,T2A]{fontenc}
\usepackage[utf8]{inputenc}
\usepackage[english,russian]{babel}
\usepackage{booktabs}
\usepackage{color,colortbl}
%\usepackage{amsmath}
%\usepackage{amsfonts}
%\usepackage{amssymb}
%\usepackage{makeidx}

\definecolor{darkishgreen}{RGB}{39,203,22}
\definecolor{LightCyan}{rgb}{0.88,1,1}
\definecolor{Gray}{gray}{0.9}
\definecolor{lightRed}{RGB}{230,170,150}
\definecolor{modRed}{RGB}{230,82,90}
\definecolor{strongRed}{RGB}{230,6,6}

\usepackage[english,russian]{babel}

\title{\bfseries Лабораторная работа №1
  
Текстовые редакторы}
\date{}

\begin{document}

\maketitle

\section{Теоретические сведения}

Текстовые процессоры являются многофункциональной программой обработки текстов. Тексты и иллюстрации многих форматов могут быть импортированы непосредственно из других программ и встроены в текст документа. В результате такой процедуры они становятся частью текстового файла, продолжая при этом существовать отдельно в виде независимых файлов в формате породившей их программы. Таким образом, графики, таблицы, графические рисунки и др. объекты могут вызывать для обработки родительские программы их подготовившие. Описанные возможности реализуются благодаря механизму объектного связывания и встраивания ---OLE (Object Linking and Embeding), который поддерживается редактором.

К основным функциональным достоинствам текстовых процессоров можно отнести следующие:
\begin{itemize}
  \item возможность автоматизированного создания документов с использованием шаблонов;
  \item работа с таблицами, включающая возможность математических расчетов по таблице;
  \item редактирование сложных математических выражений с использованием Редактора формул;
  \item проверка орфографии;
  \item широкие возможности по использованию шрифтов;
  \item встроенный редактор графических примитивов и др.
\end{itemize}

\subsection{Техника работы с окнами}
Активное окно предоставляет пользователю право доступа ко всем меню и функциям для оформления и обработки текста. Активным окном является окно обрабатываемого документа. Мультиоконная организация редактора позволяет одновременно работать над несколькими документами. Переключение между окнами (редактируемыми файлами) производится через функцию Окно главного меню редактора. Пользователю предоставлена возможность изменять размеры окна и перемещать его по экрану.Пользователь может настроить под собственные нужды состав и форму расположения на экране пиктографического меню. Для этого используется последовательность вызовов Вид --- Панели инструментов.

Основные функции по оформлению текста сосредоточены в группе пиктограмм, которая называется Линейка форматирования.
Операции связанные с изменением абзацных отступов, шириной колонок и установкой позиций табулятора производятся с использованием Координатной линейки. Для установки левостороннего или правостороннего абзацного отступа, отличного от стандартного следует отбуксировать мышью фиксатор левого или правого отступа в соответствующее место. Фиксаторы отступов расположены на координатной линейке и имеют вид треугольников. Левый фиксатор состоит из двух частей: фиксатора красной строки (верхний) и фиксатора отступа (нижний). Точная установка отступов производится в окне Абзац (меню Формат). Левое, правое, верхнее и нижнее поля документа устанавливаются путем буксировки мышью соответствующих ползунков в режиме Разметка страницы (меню Вид). В меню Вид сосредоточены все операции и установки, позволяющие определить внешний вид экрана и способа отображения на нем редактируемого документа.

\subsection{Меню вставка}
Используя операции этого меню, можно выполнять следующие основные операции: деление на страницы в ручном режиме, расстановку номеров страниц, а так же вставлять в редактируемый документ дату и время, специальные символы, файл, рамку (Кадр), иллюстрацию (Рисунок\dots), объект.

С помощью операции Кадр в активном документе можно вставить рамку, в которую будет помещаться текст, рисунок, таблица и д.т., при этом операция вставки не приводит к переформатированию всего редактируемого документа. Директива Объект используется для вставки объектов из других прикладных программ. После активизации этой директивы появляется диалоговое окно, в котором выбирается тип объекта. Список объектов может включать Equation (Редактор формул), Excel, Graph, Draw, рисунок Paint. После выбора типа объекта и нажатия ОК запускается приложение, которое создает объекты выбранного типа. Пользователь, работая в приложении, строит требуемый объект. Завершение работы с приложением сопровождается вставкой созданного объекта в документ.

\subsection{Редактор формул}
Редактор Формул может запускаться либо как самостоятельная программа, либо из тектового процессора. Для запуска Редактора Формул из редактора необходимо предварительно поместить курсор в рабочем окне в то место, куда необходимо вставить формулу. Далее необходимо вызвать последовательность директив Вставить --- Объект \dots В активизированном диалоговом окне необходимо выбрать элемент Equation (Редактор формул). После запуска редактора формул открывается его прикладное окно (обычное окно приложения), аналогичное по строению основному окну редактора. Здесь появляется возможность использования палитры символов и палитру шаблонов.
Если Редактор формул был запущен внутри редактора, то для завершения работы с ним необходимо щелкнуть левой клавишей мыши в любом месте текста главного окна.

\subsection{Работа с таблицами}
В строке пиктографического меню есть пиктограммы и для изготовления таблиц. Для создания таблицы следует поместить курсор в место будущего расположения таблицы. На экране появится прототип таблицы, перемещая мышь по которому можно определить размер таблицы, т.е. количество колонок и строк. Перемещая курсор мыши при нажатой левой кнопке, можно выбирать размер таблицы. Он динамически отображается в нижней строке прототипа. Если курсор мыши при нажатой кнопке выдвинуть за пределы сетки прототипа, то сетка увеличивается в размерах. При отпускании кнопки мыши размер таблицы фиксируется, она вставляется в документ и отображается на экране. Все ячейки созданной таблицы пусты и имеют одинаковый размер. Разделительные линии между колонками и строками в таблице отсутствуют (на печати). На экране индицируются только пунктирные линии, служащие для ориентировки и на печать не выводимые.

Редактор позволяет выполнять следующие операции с таблицами: добавление строки в таблицу, добавление колонки в таблицу, изменение ширины колонки, изменение высоты строки, изменение расстояния между колонками, сортировка таблицы, сортировка колонки, добавление ячеек, разделение и соединение ячеек, удаление ячеек, строк и колонок, разделение таблицы, выполнение математических расчетов.
\begin{enumerate}
  \item Изменение ширины колонки.
        Изменить ширину колонки можно пользуясь мышью. Для этого следует поместить указатель мыши на правой ограничительной линии изменяемой колонки. При "точном попадании" на разделительную линию указатель мыши изменяет свой вид. С помощью такого указателя разделительная линия буксируется в нужную сторону до получения необходимой ширины колонки.
  \item Изменение высоты строки.
        Для изменения высоты строки следует поступить следующим образом. Выделите строку, высоту которой вы хотели бы изменить. Выберите директиву Высота и ширина ячейки ... меню Таблица. В появившемся диалоговом окне можно установить высоту строки. Например, в позиции Высота строки установите способ установки - Точно, в позиции сколько установите нужный размер строки в сантиметрах.
  \item Разделение и соединение ячеек.
        Часто при построении таблицы появляется необходимость снабдить несколько столбцов одним общим заголовком. Для этого можно объединить несколько клеток строки для получения ячейки большей величины. С целью слияния ячеек нужно выделить все подлежащие объединению ячейки строки и вызвать директиву Объединить ячейки меню Таблица. Если требуется разделить ячейки используется директива Разбить ячейки.
  \item Вычисление суммы.
        Для вычисления суммы значений в колонках следует поступить следующим образом:
        \begin{itemize}
          \item установить курсор на ячейку, где должно быть записано итоговое значение;
          \item вызвать последовательность директив Таблица-Формула \dots
          \item  в появившемся диалоговом окне Формула в строке Формула записать выражение вида =SUM([RnCc/RmCd/]), где n,m - номера строк, c,d - номера колонок таблицы. Например, если записать =SUM([R2C5:R3C6]), тогда суммируются ячейки от Строки2 Колонкиб до СтрокиЗ Колонки 6.
        \end{itemize}
\end{enumerate}

\section{Задание на лабораторную работу}

\begin{enumerate}
  \item Запустите редактор
  \item Создайте файл, который содержит заявление о приеме на работу.
  \item Заявление должно включать:
  \begin{itemize}
    \item Должность, звание и Ф.И.О. руководителя предприятия;
	\item Ф.И.О. заявителя, его адрес и данные паспорта;
	\item текст заявления;
	\item поле подписи заявителя и дату составления заявления.
  \end{itemize}
  \item После завершения набора выделите текст заявления и измените тип и размер шрифта, выполните выравнивание правой границы текста.
  \item C использованием редактора формул наберите 3 --- 4 математических выражения из учебника математики в соответствии с вариантом из таблицы~\ref{tab:math}.
  \item Создайте таблицу и выполните необходимые вычисления подставив свои значения в стобце число согласно варианту из таблицы~\ref{tab:variant}:

\newcommand{\specialcell}[2][c]{  \begin{tabular}[#1]{@{}c@{}}#2\end{tabular}}
\begin{table}[h]
\begin{tabular}{|c|c|c|c|c|}
  \hline
  & Число & Модуль числа & Челая часть числа & \specialcell{Округление до 2 \\знаков после запятой} \\
  \hline
& 1 & ? & ? & ? \\
& 2 & ? & ? & ? \\
& 3 & ? & ? & ? \\
\hline
\specialcell{Сумма значений\\в списке} & ? & ? & ? & ? \\
\hline
\specialcell{Произведение\\значений в списке} & ? & ? & ? & ? \\
\hline
\specialcell{Наибольшее\\значение в списке} & ? & ? & ? & ? \\
\hline
\specialcell{Наименьшее\\значение в списке} & ? & ? & ? & ? \\
\hline
\specialcell{Среднее значение\\чисел в списке} & ? & ? & ? & ? \\
\hline
\specialcell{Количество\\элементов списка} & ? & ? & ? & ? \\
\hline
\end{tabular}
%\end{minipage}
\end{table}
        
  \item Оформите отчет по лабораторной работе
\end{enumerate}

\subsection{Содержание отчета}
Отчет должен быть выполнен на компьютере и сохранен в систему электронного обучения ee.istu.ru.

\noindent Отчет должен содержать:
\begin{itemize}
        \item титульный лист;
        \item цель работы;
        \item краткие теоретические сведения по теме лабораторной работы;
        \item выполненное задание;
        \item краткий вывод о проделанной работе.
\end{itemize}

\begin{table}[h]
      \caption{Варианты для формул}
      \begin{center}\label{tab:math}
        %\toprule
      \begin{tabular}{|c|c|}
        \hline
        Вариант & Тема \\
        \hline
        1 & Теория пределов\\
        \hline
        2 & Непрерывность функции\\
        \hline
        3 & Производная\\
        \hline
        4 & Дифференциал функции\\
        \hline
        5 & Производные элементарных функций\\
        \hline
        6 & Неявные функции\\
        \hline
        7 & Неопределенный интеграл\\
        \hline
        8 & Определенный интеграл\\
        \hline
        9 & Двойной интеграл\\
        \hline
        10 & Ряды Тейлора\\
        \hline
        11 & Интерполирование по Лагранжу\\
        \hline
        12 & Интерполирование по Ньютону\\
        \hline
        13 & Метод наименьших квадратов\\
        \hline
        14 & Метод Рунге-Кутта\\
        \hline
        15 & Операции над матрицами\\
        \hline
        16 & Скалярное и векторное произведение векторов\\
        \hline
      \end{tabular}
    \end{center}
\end{table}

\begin{table}[h]
      \caption{Варианты заданий}
      \begin{center}\label{tab:variant}
        %\toprule
      \begin{tabular}{|c|ccc|}
        \hline
        Вариант & 1 & 2 & 3 \\
        \hline
        1 & 2 & 3.32 & 6.004\\
        \hline
        2 & 3 & 12.64 & 6.002\\
        \hline
        3 & 4 & 14.65 & 4.002\\
        \hline
        4 & 3 & 15.32 & 4.0019\\
        \hline
        5 & 5 & 44.12 & 10.003\\
        \hline
        6 & 6 & 8.21 & 33.001\\
        \hline
        7 & 6 & 12.21 & 4.001\\
        \hline
        8 & 7 & 56.22 & 5.0012\\
        \hline
        9 & 9 & 23.62 & 67.002\\
        \hline
        10 & 4 & 45.32 & 9.003 \\
        \hline
        11 & 3 & 32.32 & 8.002 \\
        \hline
        12 & 3 & 64.16 & 10.005\\
        \hline
        13 & 4 & 16.32 & 10.002\\
        \hline
        14 & 7 & 23.24 & 2.001\\
        \hline
        15 & 5 & 54.40 & 7.005\\
        \hline
        16 & 6 & 41.23 & 7.003\\
        \hline
      \end{tabular}
    \end{center}
\end{table}

\newpage
\begin{thebibliography}{3}
  \bibitem{iv1}
    Хлебников, А.А. Информационные технологии: Учебник/ Хлебников А.А. - М.: КНОРУС, 2014.- 472 с.- (Бакалавриат).
  \bibitem{bd2}
Цветкова А.В. Информатика и информационные технологии [Электронный ресурс]: учебное пособие/ Цветкова А.В.— Электрон. текстовые данные.— Саратов: Научная книга, 2012.-182с. Режим доступа: http://www.iprbookshop.ru/6276 - ЭБС «IPRbooks»

\end{thebibliography}
\end{document}
