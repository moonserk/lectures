\documentclass[a4paper]{article}
\usepackage[T1,T2A]{fontenc}
\usepackage[utf8]{inputenc}
\usepackage[english,russian]{babel}
\usepackage{booktabs}
\usepackage{color,colortbl}
%\usepackage{amsmath}
%\usepackage{amsfonts}
%\usepackage{amssymb}
%\usepackage{makeidx}

\definecolor{darkishgreen}{RGB}{39,203,22}
\definecolor{LightCyan}{rgb}{0.88,1,1}
\definecolor{Gray}{gray}{0.9}
\definecolor{lightRed}{RGB}{230,170,150}
\definecolor{modRed}{RGB}{230,82,90}
\definecolor{strongRed}{RGB}{230,6,6}

\usepackage[english,russian]{babel}

\title{\bfseries Лабораторная работа №3

Средства по созданию презентаций и обработки графики.}
\date{}

\begin{document}

\maketitle

\noindent\textbf{Цель работы: }ознакомиться с функциональными возможностями создания презентаций

\section{Теоретические сведения}

Программа подготовки презентаций — компьютерная программа, используемая для создания, редактирования и показа презентаций на проекторе или большом экране. Программа подготовки презентаций позволяют создавать слайды (кадры) презентации и наполнять их содержимым, настраивать внешний вид презентации и возможные визуальные эффекты. Создаваемая презентация может включать в себя элементы интерактивности, такие как кнопки для перемещения между слайдами и ссылки на веб-страницы. Современные программы для создания мультимедийных презентаций позволяют использовать в презентации не только текстовые и графические изображения, но и аудио--- и видеоэлементы.


\subsection{Использование Мастера автосодержания}
С помощью Мастера автосодержания можно создать презентацию бизнес-плана, продажи новых изделий, учебного курса и много другого. В зависимости от того, что нужно сделать, Мастер выберет шаблон, который нужен, даст возможность указать желательный тип но­сителя и создаст стандартную презентацию, которую можно в дальнейшем настроить. В данном случае нужно выбрать пункт «Презентация на экране», т.е. будет создана презентация, которая будет демонстрироваться на экране компьютера. После запуска программы  выберите Мастер автосодержания. Следуйте инструкциям, появляющимся в диалоговых окнах Мастера. Щелкайте Далее, чтобы перейти от одного диалогового окна к следующему. Вы можете вернуться к предыдущим диалоговым окнам в любое время в течение этого процесса, щел­кая кнопку Назад. Кнопки Добавить и Удалить на этапе выбора Вида презентации не пригодятся. Когда перейдете к последнему диалоговому окну, нажмите кнопку Готово. Мастер создаст слайд --- шоу и отобразит его в Обычном режиме (Каждое слайд --- шоу содержит инструкции, которые нужно заменить собственным текстом). С левой стороны экрана содержится панель Структура/Слайды, которая дает возможность увидеть схему презентации или эскизы каждого отдельного слайда. Чтобы увидеть определенный слайд, просто щелкните заголовок слайда или его эскиз.

\subsection{Создание презентации с помощью шаблона оформления}
При создании презентации с помощью шаблона Мастер не участвует. После того, как будет запущена программа, выберите пункт Из шаблона оформления в области задач Создание презентации и выберите эскиз подходящего дизайна из списка Применить шаблон оформления. После выбора пункта Из шаблона оформления слева появится область задач. С помощью полосы прокрутки просмотрите все эскизы шаблонов оформления и, выбрав нужный, щелкните на нем мышью. После того как выбран эскиз дизайна, программа создает презентацию из одного слайда с заданными по умолчанию структурой, оформлением и цветовой схемой. Справа повится область задач Дизайн слайда, в которой можно отредактировать все слайд-шоу.

Далее, выбрав пункт Цветовые схемы или Эффекты анимации, можно выбрать подходящий шаблон оформления, цветовую схему или анимацию из списка в нижней части окна. После настройки внешнего вида слайдов можно начать добавлять новые слайды. Для этого откройте меню Вставка и выберете Создать слайд или нажмите кнопку Создать слайд на панели инструментов. Новый слайд появится сразу после текущего, поэтому предварительно нужно выделить тот слайд, после которого должен появиться новый. После добавления нового слайда справа появится панель задач Разметка слайда, с помощью которой можно выбрать структуру слайда. Готовые структуры слайдов содержат текстовые поля для заголовков и списков, поля для вставки графики и диаграмм и дополнительные объекты, которые можно изменить, чтобы создать слайд. Выберите подходящую структуру, чтобы применить ее к новому слайду. Программа установит для новых слайдов тот же самый фон и цвет, как и для всех остальных слайдов в презентации. Вернуться к панели задач Дизайн слайда можно, нажав на кнопку Конструктор на панели инструментов.н

\section{Задание на лабораторную работу}

Подготовить презентацию фирмы, её продуктов и услуг одним из описанных способов. В презентации обязательно должны быть представлены образцы продукции, их описание, сравнение с аналогичными товарами других фирм. Не менее 5 сдайдов.

\subsection{Содержание отчета}
Отчет должен быть выполнен на компьютере и сохранен в систему электронного обучения ee.istu.ru.

\noindent Отчет должен содержать:
\begin{itemize}
        \item титульный лист;
        \item цель работы;
        \item краткие теоретические сведения по теме лабораторной работы;
        \item выполненное задание;
        \item краткий вывод о проделанной работе.
\end{itemize}

\begin{thebibliography}{3}
  \bibitem{iv1}
    Хлебников, А.А. Информационные технологии: Учебник/ Хлебников А.А. - М.: КНОРУС, 2014.- 472 с.- (Бакалавриат).
  \bibitem{bd2}
Цветкова А.В. Информатика и информационные технологии [Электронный ресурс]: учебное пособие/ Цветкова А.В.— Электрон. текстовые данные.— Саратов: Научная книга, 2012.-182с. Режим доступа: http://www.iprbookshop.ru/6276 - ЭБС «IPRbooks»

\end{thebibliography}

\end{document}
