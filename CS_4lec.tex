\documentclass[a4paper]{article}
\usepackage[T1,T2A]{fontenc}
\usepackage[utf8]{inputenc}
\usepackage[english,russian]{babel}
\usepackage{booktabs}
\usepackage{color,colortbl}
%\usepackage{amsmath}
%\usepackage{amsfonts}
%\usepackage{amssymb}
%\usepackage{makeidx}

\definecolor{darkishgreen}{RGB}{39,203,22}
\definecolor{LightCyan}{rgb}{0.88,1,1}
\definecolor{Gray}{gray}{0.9}
\definecolor{lightRed}{RGB}{230,170,150}
\definecolor{modRed}{RGB}{230,82,90}
\definecolor{strongRed}{RGB}{230,6,6}

\usepackage[english,russian]{babel}

\begin{document}


\section{Представление данных в ЭВМ}

\subsection{Кодирование информации}
 В процессе преобразования информации из одной формы представления (знаковой системы) в другую осуществляется кодирование. Средством кодирования служит таблица соответствия, которая устанавливает взаимно однозначное соответствие между знаками или группами знаков двух различных знаковых систем.В процессе обмена информацией часто приходится производить операции кодирования и декодирования информации. При вводе знака алфавита в компьютер путем нажатия соответствующей клавиши на клавиатуре выполняется его кодирование, т. е. преобразование в компьютерный код. При выводе знака на экран монитора или принтер происходит обратный процесс — декодирование, когда из компьютерного кода знак преобразуется в графическое изображение.
     
 \subsubsection{Кодирование изображений и звука}
 Информация, в том числе графическая и звуковая, может быть представлена в аналоговой или дискретной форме. При аналоговом представлении физическая величина принимает бесконечное множество значений, причем ее значения изменяются непрерывно. При дискретном представлении физическая величина принимает конечное множество значений, причем ее величина изменяется скачкообразно.

 Примером аналогового представления графической информации может служить, скажем, живописное полотно, цвет которого изменяется непрерывно, а дискретного — изображение, напечатанное с помощью струйного принтера и состоящее из отдельных точек разного цвета.Примером аналогового хранения звуковой информации является виниловая пластинка (звуковая дорожка изменяет свою форму непрерывно), а дискретного — аудиокомпакт-диск (звуковая дорожка которого содержит участки с различной отражающей способностью).

 Графическая и звуковая информация из аналоговой формы в дискретную преобразуется путем дискретизации, т. е. разбиения непрерывного графического изображения и непрерывного (аналогового) звукового сигнала на отдельные элементы. В процессе дискретизации производится кодирование, т. е. присвоение каждому элементу конкретного значения в форме кода.

 Дискретизация — это преобразование непрерывных изображений и звука в набор дискретных значений, каждому из которых присваивается значение его кода.

 Кодирование информации в живых организмах. Генетическая информация определяет строение и развитие живых организмов и передается по наследству. Хранится генетическая информация в клетках организмов в структуре молекул ДНК (дезоксирибонуклеиновой кислоты). Молекулы ДНК состоят из четырех различных составляющих (нуклеотидов), которые образуют генетический алфавит. Молекула ДНК человека включает в себя около трех миллиардов пар нуклеотидов, и в ней закодирована вся информация об организме человека: его внешность, здоровье или предрасположенность к болезням, способности и т. д.

\subsection{Двоичная система счисления}
Система счисления — это знаковая система, в которой числа записываются по определенным правилам с помощью цифр — символов некоторого алфавита. Например, в десятичной системе для записи числа существует десять всем хорошо известных цифр: О, 1, 2 и т. д.

Все системы счисления делятся на позиционные и непозиционные. В позиционных системах счисления значение цифры зависит от ее положения в записи числа, а в непозиционных — не зависит. Позиция цифры в числе называется разрядом. Разряд числа возрастает справа налево, от младших разрядов к старшим. В непозиционных системах вес цифры (т.е. тот вклад, который она вносит в значение числа) не зависит от ее позиции в записи числа. Так, в римской системе счисления в числе ХХХII (тридцать два) вес цифры Х в любой позиции равен просто десяти.В позиционных системах счисления вес каждой цифры изменяется в зависимости от ее положения (позиции) в последовательности цифр, изображающих число. Например, в числе 757,7 первая семерка означает 7 сотен, вторая – 7 единиц, а третья – 7 десятых долей единицы. Сама же запись числа 757,7 означает сокращенную запись выражения:
\begin{equation}
  700 + 50 + 7 + 0,7 = 7*10_{2} + 5*10_{1} + 7*10_{0} + 7*10_{-1} = 757,7
\end{equation}

Любая позиционная система счисления характеризуется своим основанием. Основание позиционной системы счисления — это количество различных знаков или символов, используемых для изображения цифр в данной системе. За основание системы можно принять любое натуральное число — два, три, четыре и т.д. Следовательно, возможно бесчисленное множество позиционных систем: двоичная, троичная, четверичная и т.д. Запись чисел в каждой из систем счисления с основанием q означает сокращенную запись выражения:
\begin{equation}
   a_{n-1} * q^{n-1} + a_{n-2}* q^{n-2} + \cdots + a_{1}* q^{1} + a_{0}*q^{0} + a_{-1}*q^{-1} + \cdots + a_{-m} * q^{-m}
\end{equation}
\begin{center}
  где $a_{i}$ – цифры системы счисления; $n$ и $m$ – число целых и дробных разрядов, соответственно.
\end{center}
Каждая позиционная система использует определенный алфавит цифр и основание. В позиционных системах счисления основание системы равно количеству цифр (знаков в ее алфавите) и определяет, во сколько раз различаются значения цифр соседних разрядов числа.
Кроме десятичной широко используются системы с основанием, являющимся целой степенью числа 2, а именно:
\begin{itemize}
  \item двоичная (используются цифры 0, 1)
  \item восьмеричная (используются цифры 0, 1, $\ldots$, 7)
  \item шестнадцатеричная (для первых целых чисел от нуля до девяти используются цифры 0, 1, $\ldots$, 9, а для следующих чисел --- от десяти до пятнадцати --- в качестве цифр используются символы A, B, C, D, E, F).
\end{itemize}
% В таблице 1 указаны первые шестнадцать чисел этих систем счисления.
\begin{table}\label{table:nums}
\caption{Запись первых шестнадцати целых чисел в разных системах счисления}
      \begin{center}
      \begin{tabular}{c * {4}{c}}
        \toprule
        двоичная & восьмеричная & десятичная & шестнадцатеричная \\
        \toprule
        00000 & 0 & 0 & 0 \\
        00001 & 1 & 1 & 1 \\
        00010 & 2 & 2 & 2 \\
        00011 & 3 & 3 & 3 \\
        00100 & 4 & 4 & 4 \\
        00101 & 5 & 5 & 5 \\
        00110 & 6 & 6 & 6 \\
        00111 & 7 & 7 & 7 \\
        01000 & 10 & 8 & 8 \\
        01001 & 11 & 9 & 9 \\
        01010 & 12 & 10 & A \\
        01011 & 13 & 11 & B \\
        01100 & 14 & 12 & C \\
        01101 & 15 & 13 & D \\
        01110 & 16 & 14 & E \\
        01111 & 17 & 15 & F \\
        10000 & 20 & 16 & 10 \\
        \bottomrule
      \end{tabular}
      \end{center}
    \end{table}

Из всех систем счисления особенно проста и поэтому интересна для технической реализации в компьютерах двоичная система счисления.

Рассмотрим в качестве примера десятичное число 555. Цифра 5 встречается трижды, причем самая правая обозначает пять единиц, вторая справа — пять десятков и, наконец, третья — пять сотен. Число 555 записано в привычной для нас свернутой форме. Мы настолько привыкли к такой форме записи, что уже не замечаем, как в уме умножаем цифры числа на различные степени числа 10. В развернутой форме запись числа 555 в десятичной системе выглядит следующим образом:
$$555_{10} = 5 * 10^{2} + 5 * 10^{1} + 5 * 10^{0}$$
Как видно из примера, число в позиционных системах счисления записывается в виде суммы степеней основания (в данном случае 10), коэффициентами при этом являются цифры данного числа. В двоичной системе основание равно 2, а алфавит состоит из двух цифр (0 и 1). В развернутой форме двоичные числа записываются в виде суммы степеней основания 2 с коэффициентами, в качестве которых выступают цифры 0 или 1. Например, развернутая запись двоичного числа $101_{2}$ в десятичном представлении будет иметь вид:
$$1 * 2^{2} + 0 * 2^{1} + 1 * 2^{0}$$

\subsection{Выполнение арифметических операций в двоичной системе счисления}

\subsubsection{Сложение}
В основе сложения чисел в двоичной системе счисления лежит таблица сложения одноразрядных двоичных чисел (Таблица 2). Важно обратить внимание на то, что при сложении двух единиц производится перенос в старший разряд. Это происходит тогда, когда величина числа становится равной или большей основания системы
\begin{table}
  \caption{Таблица сложения одноразрядных двоичных чисел}
  \begin{center}
  $0 + 0 = 0$\\
  $0 + 1 = 1$\\
  $1 + 0 = 1$\\
  $1 + 1 = 10$\\
  \end{center}
\end{table}

\subsubsection{Вычитание}В основе вычитания двоичных чисел лежит таблица вычитания одноразрядных двоичных чисел (Таблица 3).
\begin{table}
  \caption{Таблица вычитания одноразрядных двоичных чисел}
  \begin{center}
  $0 - 0 = 0$\\
  $0 - 1 = -11$\\
  $1 - 0 = 1$\\
  $1 - 1 = 0$\\
  \end{center}
\end{table}
    
\subsubsection{Умножение}
В основе умножения лежит таблица умножения одноразрядных двоичных чисел (Таблица 4).
\begin{table}
  \caption{Таблица умножения одноразрядных двоичных чисел}
  \begin{center}
  $0 * 0 = 0$ \\
  $0 * 1 = 0$ \\
  $1 * 0 = 0$ \\
  $1 * 1 = 1$ \\
  \end{center}
\end{table}
Умножение многоразрядных двоичных чисел осуществляется в соответствии с этой таблицей умножения по обычной схеме, применяемой в десятичной системе счисления, с последовательным умножением множимого на очередную цифру множителя.

\subsection{Числа со знаком и операции с ними}
Прямым кодом числа называется двоичное представление его модуля, т. е. прямой код используется для представления положительных чисел. В прямом коде удобно выполнять операции сложения (Таблица 5), но не вычитания (Таблица 6). В случае заема из бита числа А вычитается бит заема, а затем бит числа В. Как видно из результата, бит знака равен нулю, а значит, мы получили положительное число. Как видно, операцию вычитания выполнять неудобно (легко запутаться). Поэтому, в ЭВМ вычитание заменяется сложением чисел в обратном или дополнительном коде, что позволяет и для сложения, и для вычитания использовать одно и то же устройство $-$ сумматор.
\begin{table}\label{table:add}
      \caption{Сложение чисел 43 и 29 в двоичной системе счисления}
      \begin{center}
      \begin{tabular}{c * {11}{c}}
        \toprule
        Номер бита & Перенос 8 & 7 & 6 & 5 & 4 & 3 & 2 & 1 & 0 \\
        \toprule
        Число А ($43 = 32 + 8 + 2 + 1$) &  & 0 & 0 & 1 & 0 & 1 & 0 & 1 & 1\\
        \midrule
        Число B ($29 = 16 + 8 + 4 + 1$) &  & 0 & 0 & 0 & 1 & 1 & 1 & 0 & 1\\
        \midrule
        Перенос из разряда &  &  & 1 & 1 & 1 & 1 & 1 & 1 & \\
        \midrule
        Сумма ($72 = 43 + 29$) &  & 0 & 1 & 0 & 0 & 1 & 0 & 0 & 0 \\
        \bottomrule
      \end{tabular}
    \end{center}
\end{table}


\begin{table}\label{table:sub}
      \caption{Вычитание  чисел 43 и 29 в двоичной системе счисления}
      \begin{center}
      \begin{tabular}{c * {11} {c}}
        \toprule
        Номер бита & Перенос 8 & 7 & 6 & 5 & 4 & 3 & 2 & 1 & 0 \\
        \toprule
        Число А ($43 = 32 + 8 + 2 + 1$) &  & 0 & 0 & 1 & 0 & 1 & 0 & 1 & 1\\
        \midrule
        Число B ($29 = 16 + 8 + 4 + 1$) &  & 0 & 0 & 0 & 1 & 1 & 1 & 0 & 1\\
        \midrule
        Перенос из разряда &  &  &  & 1 & 1 & 1 &  &  & \\
        \midrule
        Сумма ($14 = 43 - 29$) &  & 0 & 0 & 0 & 0 & 1 & 1 & 1 & 0 \\
        \midrule
      \end{tabular}
      \end{center}
    \end{table}

\subsubsection{Числа со знаком}

Для представления чисел со знаком используется обратный или дополнительный код. Для положительных чисел он совпадает с прямым. Старший бит (до разряда переноса определяет знак числа 0 для положительных и 1 для отрицательных). В операциях сложения и вычитания разрядность чисел должна быть одинаковой.

\subsubsection{Обратный код}
Обратным кодом отрицательного числа называется инверсия разрядов этого числа (замена 0 на 1, а 1 на 0). Для положительных чисел обратный код совпадает с прямым (можно считать, что для положительных чисел его не существует). После операции сложения в обратном коде получается промежуточный результат, к которому нужно прибавить значение разряда переноса, после чего отбросить этот разряд.

\begin{table}\label{tabel:sub_OB}
      \caption{Вычитание чисел 43 и 29 в двоичной системе счисления с использованрием обратного кода}
      \begin{center}
        %\toprule
      \begin{tabular}{c * {11}{c}}
        \toprule
        Номер бита & Перенос 8 & 7 & 6 & 5 & 4 & 3 & 2 & 1 & 0 \\
        \toprule
        Модуль числа 29 ($29 = 16 + 8 + 4 + 1$) &  & 0 & 0 & 0 & 1 & 1 & 1 & 0 & 1\\
        \midrule
        Обратый код числа 29 ($29 = 16 + 8 + 4 + 1$) &  & 1 & 1 & 1 & 0 & 0 & 0 & 1 & 0\\
        \midrule
        Число А ($43 = 32 + 8 + 2 + 1$) &  & 0 & 0 & 1 & 0 & 1 & 0 & 1 & 1\\
        \midrule
        Перенос из разряда &  &  & 1 &  &  &  & 1 &  & \\
        \midrule
        Сумма (промежуточный результат) & 1 & 0 & 0 & 0 & 0 & 1 & 1 & 0 & 1 \\
        \midrule
        + перенос &  &  &  &  &  &  &  &  & 1 \\
        \midrule
        Сумма (окончательный результат) &  & 0 & 0 & 0 & 0 & 1 & 1 & 1 & 0 \\
        \bottomrule
      \end{tabular}
    \end{center}
\end{table}

\subsubsection{Дополнительный код}
В дополнительном коде в отличие от  обратного возможный перенос учитывается в самих числах, а не в результате. Для получения дополнительного кода числа нужно к обратному коду прибавить 1. Для положительных чисел дополнительный код совпадает с прямым (можно считать, что для положительных чисел его не существует). В большинстве современных вычислительных устройств используется не обратный, а дополнительный код. Если к n-разрядному положительному числу Х добавить дополнительный код числа (-Х), получим число $2^{n}$, т. е. данный код 'дополняет' Х до нуля (если отбросить старший бит в числе $2^{n}$ --- единицу переноса). После операции сложения в дополнительном коде получается промежуточный результат, от которого нужно отбросить разряд переноса (Таблица 8).


\begin{table}\label{tabel:sub_DP}
      \caption{Вычитание чисел 43 и 29 в двоичной системе счисления с использованрием дополнительного кода}
      \begin{center}
        %\toprule
      \begin{tabular}{c * {11}{c}}
        \toprule
        Номер бита & Перенос 8 & 7 & 6 & 5 & 4 & 3 & 2 & 1 & 0 \\
        \toprule
        Модуль числа 29 ($29 = 16 + 8 + 4 + 1$) &  & 0 & 0 & 0 & 1 & 1 & 1 & 0 & 1\\
        \midrule
        Обратый код числа 29 ($29 = 16 + 8 + 4 + 1$) &  & 1 & 1 & 1 & 0 & 0 & 0 & 1 & 0\\
        \midrule
        +1 ($29 = 16 + 8 + 4 + 1$) &  &  &  &  &  &  &  &  & 1\\
        \midrule
        Дополнительный код числа 29 ($29 = 16 + 8 + 4 + 1$) &  & 1 & 1 & 1 & 0 & 0 & 0 & 1 & 1\\
        \midrule
        Число А ($43 = 32 + 8 + 2 + 1$) &  & 0 & 0 & 1 & 0 & 1 & 0 & 1 & 1\\
        \midrule
        Перенос из разряда &  & 1 & 1 &  &  &  & 1 & 1 & \\
        \midrule
        Сумма (промежуточный результат) & 1 & 0 & 0 & 0 & 0 & 1 & 1 & 1 & 0 \\
        \midrule
        + перенос &  &  &  &  &  &  &  &  & 1 \\
        \midrule
        Сумма (окончательный результат) &  & 0 & 0 & 0 & 0 & 1 & 1 & 1 & 0 \\
        \bottomrule
      \end{tabular}
    \end{center}
\end{table}

\subsubsection{Дополнительный модифицированный код}

В модифицированном коде для знака используется 2 бита. Значение этих разрядов должно совпадать. Разряд переноса в результате операции отбрасывается. Если в результате операции сложения в модифицированном коде получаются разные значения знаковых разрядов, то старший знаковый бит нужно продублировать еще раз, а младший знаковый бит отнести к значению числа (разрядность увеличится на 1 бит). Рассмотрим возможные варианты операций для чисел А и В. Исходные операнды занимают 6 бит + 2 бита для знаков (Таблица 9).

\begin{table}\label{table:sub}
  \caption{Сложение чисел 43 и 29 в двоичной системе с использованием дополнительного модифицированного кода}
      \begin{center}
      \begin{tabular}{c * {11} {c}}
        \toprule
        Номер бита & Перенос 8 & \cellcolor{lightRed} Знак 7 & \cellcolor{lightRed}Знак 6 & 5 & 4 & 3 & 2 & 1 & 0 \\
        \toprule
        Число А ($43 = 32 + 8 + 2 + 1$) &  & \cellcolor{lightRed} 0 & \cellcolor{lightRed} 0 & 1 & 0 & 1 & 0 & 1 & 1\\
        \midrule
        Число B ($29 = 16 + 8 + 4 + 1$) &  & \cellcolor{lightRed} 0 & \cellcolor{lightRed} 0 & 0 & 1 & 1 & 1 & 0 & 1\\
        \midrule
        Перенос из разряда &  &  & 1 & 1 & 1 & 1 & 1 & 1 & \\
        \midrule
        Промежуточный результат &  &\cellcolor{lightRed} 0 & \cellcolor{lightRed}1 & 0 & 0 & 1 & 0 & 0 & 0\\
        \midrule
        Сумма ($14 = 43 - 29$) & \cellcolor{lightRed} 0 & \cellcolor{lightRed} 0 & 1 & 0 & 0 & 1 & 0 & 0 & 0 \\
        \midrule
      \end{tabular}
      \end{center}
    \end{table}


\subsection{Представление вещественных чисел в компьютере}

Вещественными числами в компьютерной технике называются числа, имеющие дробную часть. При их написании вместо запятой принято писать точку. Так, например, число 5 — целое, а числа 5.1 и 5.0 — вещественные. Для удобства отображения чисел, принимающих значения из достаточно широкого диапазона (то есть, как очень маленьких, так и очень больших), используется форма записи чисел с порядком основания системы счисления. Например, десятичное число 1.25 можно в этой форме представить так:
\[1.25*10^{0} = 0.125*10^{1} = 0.0125*10^{2} = \ldots ,\]
или так:
\[12.5*10^{-1} = 125.0*10^{-2} = 1250.0*10^{-3} = \ldots \]
Любое число N в системе счисления с основанием q можно записать в виде $N = M * q^{p}$, где M называется мант



иссой числа, а p — порядком. Такой способ записи чисел называется представлением с плавающей точкой.
Если 'плавающая' точка расположена в мантиссе перед первой значащей цифрой, то при фиксированном количестве разрядов, отведённых под мантиссу, обеспечивается запись максимального количества значащих цифр числа, то есть максимальная точность представления числа в машине. Из этого следует:\newline Мантисса должна быть правильной дробью, первая цифра которой отлична от нуля: M из [0.1, 1]. Такое, наиболее выгодное для компьютера, представление вещественных чисел называется нормализованным.cМантиссу и порядок q-ичного числа принято записывать в системе с основанием q, а само основание --- x в десятичной системе.
Примеры нормализованного представления: \newline
Десятичная система: \newline
$753.15 = 0.75315*10^{3}$;
$-0.000034 = -0.34*10^{-4}$ \newline
Двоичная система: \newline
$-101.01 = -0.10101*2^{11} (порядок 11_{2} = 3_{10})$;
$-0.000011 = 0.11*2^{-100} (порядок -100_{2} = -4_{10})$ \newline
Вещественные числа в компьютерах различных типов записываются по-разному. При этом компьютер обычно предоставляет программисту возможность выбора из нескольких числовых форматов наиболее подходящего для конкретной задачи — с использованием четырех, шести, восьми или десяти байтов.
%В качестве примера приведем характеристики форматов вещественных чисел, используемых IBM-совместимыми персональными компьютерами:

При хранении числа с плавающей точкой отводятся разряды для мантиссы, порядка, знака числа и знака порядка:
\begin{itemize}
\item Чем больше разрядов отводится под запись мантиссы, тем выше точность представления числа.
\item Чем больше разрядов занимает порядок, тем шире диапазон от наименьшего отличного от нуля числа до наибольшего числа, представимого в машине при заданном формате.
\end{itemize}

\subsection{Выполнение арифметических действий над нормализованными числами}

К началу выполнения арифметического действия операнды операции помещаются в соответствующие регистры АЛУ (арифметико-логическое устройство).
\subsubsection{Сложение и вычитание}
При сложении и вычитании сначала производится подготовительная операция, называемая выравниванием порядков. В процессе выравнивания порядков мантисса числа с меньшим порядком сдвигается в своем регистре вправо на количество разрядов, равное разности порядков операндов. После каждого сдвига порядок увеличивается на единицу. В результате выравнивания порядков одноименные разряды чисел оказываются расположенными в соответствующих разрядах обоих регистров, после чего мантиссы складываются или вычитаются. В случае необходимости полученный результат нормализуется путем сдвига мантиссы результата влево. После каждого сдвига влево порядок результата уменьшается на единицу.
\begin{itemize}
\item[Пример 1] Сложить двоичные нормализованные числа $0.10111 * 2^{-1}$ и $0.11011*2^{10}y$. Разность порядков слагаемых здесь равна трем, поэтому перед сложением мантисса первого числа сдвигается на три разряда вправо:

\[
\begin{array}{r}
+
\begin{array}{r}
0.00010111 * 2^{10}\\
0.11011000 * 2^{10}\\
\end{array} \\
\midrule
\begin{array}{r}
0.11101111 * 2^{10}
\end{array}
\end{array}
\]

\item[Пример 2] Выполнить вычитание двоичных нормализованных чисел $0.10101*2^{10}$ и $0.11101*2^{1}$. Разность порядков уменьшаемого и вычитаемого здесь равна единице, поэтому перед вычитанием мантисса второго числа сдвигается на один разряд вправо:

\[
\begin{array}{r}
-
\begin{array}{r}
0.101010 * 2^{10}\\
0.011101 * 2^{10}\\
\end{array} \\
\midrule
\begin{array}{r}
0.001101y * 2^{10}
\end{array}
\end{array}
\]


\end{itemize}

Результат получился не нормализованным, поэтому его мантисса сдвигается влево на два разряда с соответствующим уменьшением порядка на две единицы: $0.1101*2^{0}$.
\subsubsection{Умножение}
При умножении двух нормализованных чисел их порядки складываются, а мантиссы перемножаются.
\begin{itemize}
\item[Пример 3] Выполнить умножение двоичных нормализованных чисел:
    $$(0.11101*2^{101})*(0.1001*2^{11}) = (0.11101*0.1001)* 2^{(101+11)} = 0.100000101*2^{1000}$$
\end{itemize}
\subsubsection{Деление}
При делении двух нормализованных чисел из порядка делимого вычитается порядок делителя, а мантисса делимого делится на мантиссу делителя. Затем в случае необходимости полученный результат нормализуется.
\begin{itemize}
\item[Пример 4] Выполнить деление двоичных нормализованных чисел:
    \[0.1111*2^{100} : 0.101*2^{11} = (0.1111 : 0.101) * 2^{(100–11)} = 1.1*2^{1} = 0.11*2^{10}\]
\end{itemize}
Использование представления чисел с плавающей точкой существенно усложняет схему арифметико-логического устройства.

\subsection{Двоичное кодирование текстовой информации}

Начиная с конца 60-х годов компьютеры все больше стали использоваться для обработки текстовой информации, и в настоящее время основная доля персональных компьютеров в мире (и большая часть времени) занята обработкой именно текстовой информации. Традиционно для кодирования одного символа используется количество информации, равное 1 байту, т. е. / = 1 байт = 8 бит.
    
Если рассматривать символы как возможные события, то можно вычислить, какое количество различных символов можно закодировать:
\[N = 2^{I} = 2^{8} = 256 \]
Такое количество символов вполне достаточно для представления текстовой информации, включая прописные и заглавные буквы русского и латинского алфавита, цифры, знаки, графические символы и т. д. Кодирование заключается в том, что каждому символу ставится в соответствие уникальный десятичный код от 0 до 255 или соответствующий ему двоичный код от 00000000 до 11111111. Таким образом, человек различает символы по их начертанию, а компьютер — по их коду. При вводе в компьютер текстовой информации происходит ее двоичное кодирование, изображение символа преобразуется в его двоичный код. Пользователь нажимает на клавиатуре клавишу с символом — и в компьютер поступает определенная последовательность из восьми электрических импульсов (двоичный код символа). Код символа хранится в оперативной памяти компьютера, где занимает одну ячейку. В процессе вывода символа на экран компьютера производится обратный процесс — декодирование, т. е. преобразование кода символа в его изображение. Важно, что присвоение символу конкретного кода — это вопрос соглашения, которое фиксируется в кодовой таблице. Первые 33 кода (с 0 по 32) обозначают не символы, а операции (перевод строки, ввод пробела и т. д.). Коды с 33 по 127 — интернациональные и соответствуют символам латинского алфавита, цифрам, знакам арифметических операций и знакам препинания. Коды с 128 по 255 являются национальными, т. е. в национальных кодировках одному и тому же коду отвечают различные символы. К сожалению, в настоящее время существует пять различных кодовых таблиц для русских букв (КОИ-8, СР1251, СР866, Мае, ISO), поэтому тексты, созданные в одной кодировке, не будут правильно отображаться в другой.
Каждая кодировка задается своей собственной кодовой таблицей. Одному и тому же двоичному коду в различных кодировках поставлены в соответствие различные символы. В настоящее время широкое распространение получил новый международный стандарт Unicode, который отводит на каждый символ не один байт, а два, и потому с его помощью можно закодировать не 256 символов, а  $N = 2^{16} = 65536$ различных символов.

\end{document}
