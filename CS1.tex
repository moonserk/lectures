\documentclass[a4paper]{article}
\usepackage[T1,T2A]{fontenc}
\usepackage[utf8]{inputenc}
\usepackage[english,russian]{babel}
%\usepackage{amsmath}
%\usepackage{amsfonts}
%\usepackage{amssymb}
%\usepackage{makeidx}

\usepackage[english,russian]{babel}

\begin{document}

\section{Информатика, ее составляющие и роль в современном мире}

\subsection{Информатика, ее место в системе наук}

Термин информатика возник в 60-х гг. во Франции для названия области, занимающейся автоматизированной обработкой информации с помощью электронных вычислительных машин. Французский термин informatigue (информатика)образован путем слияния слов information (информация) и automatigue (автоматика) и означает "информационная автоматика или автоматизированная переработка информации". В англоязычных странах этому термину соответствует синоним computer science (наука о компьютерной технике).

Выделение информатики как самостоятельной области человеческой деятельности в первую очередь связано с развитием компьютерной техники. Причем основная заслуга в этом принадлежит микропроцессорной технике, появление которой в середине 70-х гг. послужило началом второй электронной революции. С этого времени элементной базой вычислительной машины становятся интегральные схемы и микропроцессоры, а область, связанная с созданием и использованием компьютеров, получила мощный импульс в своем развитии. Термин "информатика'' приобретает новое дыхание и используется не только для отображения достижений компьютерной техники, но и связывается с процессами передачи и обработки информации.

В нашей стране подобная трактовка термина 'информатика' утвердилась с момента принятия решения в 1983 г. на сессии годичного собрания Академии наук СССР об организации нового отделения информатики, вычислительной техники и автоматизации. Информатика трактовалась как 'комплексная научная и инженерная дисциплина, изучающая все аспекты разработки, проектирования, создания, оценки, функционирования основанных на ЭВМ систем переработки информации, их применения и воздействия на различные области социальной практики'. Информатика в таком понимании нацелена на разработку общих методологических принципов построения информационных моделей. Поэтому методы информатики применимы всюду, где существует возможность описания объекта, явления, процесса и т.п. с помощью информационных моделей.

Существует множество определений информатики, что связано с многогранностью ее функций, возможностей, средств и методов. Один из вариантов которые предлагает википедия: Информáтика (фр. Informatique) — наука о методах и процессах сбора, хранения, обработки, передачи, анализа и оценки информации с нием компьютерных технологий, обеспечивающих возможность её использования для принятия решений. Информатика включает дисциплины, относящиеся к обработке информации в вычислительных машинах и вычислительных сетях: как абстрактные, вроде анализа алгоритмов, так и конкретные, например, разработка языков программирования и протоколов передачи данных.

Темами исследований в информатике как науки являются вопросы: что можно, а что нельзя реализовать в программах и базах данных (теория вычислимости и искусственный интеллект), каким образом можно решать специфические вычислительные и информационные задачи с максимальной эффективностью (теория сложности вычислений), в каком виде следует хранить и восстанавливать информацию специфического вида (структуры и базы данных), как программы и люди должны взаимодействовать друг с другом (пользовательский интерфейс и языки программирования и представление знаний) и т. п.

Информатика обычно представляют состоящей из двух частей:

Теоретическая информатика

Огромное поле исследований теоретической информатики включает как классическую теорию алгоритмов, так и широкий спектр тем, связанных с более абстрактными логическими и математическими аспектами вычислений. Теоретическая информатика занимается теориями формальных языков, автоматов, алгоритмов, вычислимости и вычислительной сложности, а также вычислительной теорией графов, криптологией, логикой (включая логику высказываний и логику предикатов), формальной семантикой и закладывает теоретические основы для разработки компиляторов языков программирования.

Прикладная информатика

Прикладная информатика направлена на применение понятий и результатов теоретической информатики к решению конкретных задач в конкретных прикладных областях.

\subsection{Связь информатики с другими науками}

Информатика — очень широкая сфера научных знаний, возникшая на стыке нескольких фундаментальных и прикладных дисциплин. Фундаментальная наука — наука, изучающая объективные законы природы и общества, осуществляющая теоретическую систематизацию знаний о действительности. К фундаментальным принято относить те науки, основные понятия которых носят общенаучный характер, используются во многих других науках и видах деятельности. Как комплексная научная дисциплина информатика связана с:

\begin{itemize}
\item философией и психологией — через учение об информации и теорию познания;
\item математикой — через теорию математического моделирования, дискретную математику, математическую логику и теорию алгоритмов;
\item лингвистикой — через учение о формальных языках и о знаковых системах;
\item кибернетикой — через теорию информации и теорию управления;
\item физикой и химией, электроникой и радиотехникой — через «материальную» часть компьютера и информационных систем.
\end{itemize}
Роль информатики в развитии общества чрезвычайно велика. Она является научным фундаментом процесса информатизации общества. С ней связано прогрессивное увеличение возможностей компьютерной техники, развитие информационных сетей, создание новых информационных технологий, которые приводят к значительным изменениям во всех сферах общества: в производстве, науке, образовании, медицине и т.д.

\subsection{Экономические, социальные и правовые аспекты информационных технологий}

Роль информатики в развитии общества чрезвычайно велика. С ней связано начало революции в области накопления, передачи и обработки информации. Эта революция, следующая за революциями в овладении веществом и энергией, затрагивает и коренным образом преобразует не только сферу материального производства, но и интеллектуальную, духовную сферы жизни. Рост производства компьютерной техники, развитие информационных сетей, создание новых информационных технологий приводят к значительным изменениям во всех сферах общества: в производстве, науке, образовании, медицине и т.д.

Информатизация общества — организованный социально-экономический и научно-технический процесс создания оптимальных условий для удовлетворения информационных потребностей и реализации прав граждан, органов государственной власти, органов местного самоуправления организаций, общественных объединений на основе формирования и использования информационных ресурсов.
Цель информатизации — улучшение качества жизни людей за счет увеличения производительности и облегчения условий их труда.
Информатизация — это сложный социальный процесс, связанный со значительными изменениями в образе жизни населения. Он требует серьёзных усилий на многих направлениях, включая ликвидацию компьютерной неграмотности, формирование культуры использования новых информационных технологий и др.
Бурное развитие информационных технологий имеет и негативные аспекты:
Расслоение общества на 'продвинутых специалистов' и тех, кто не может угнаться за быстрыми изменениями условий жизни (требуется изучать все новые средства связи, программные продукты).
Вторжение в частную жизнь людей и организаций, шпионаж с помощью технических средств.

\subsubsection{Правовая охрана программ и данных.}

Правовая охрана программ для ЭВМ и баз данных впервые в полном объеме введена в Российской Федерации Законом «О правовой охране программ для электронных вычислительных машин и баз данных», который вступил в силу 20 октября 1992 г. Предоставляемая настоящим законом правовая охрана распространяется на все виды программ для компьютеров (в том числе на операционные системы и программные комплексы), которые могут быть выражены на любом языке и в любой форме. Для признания и реализации авторского права на компьютерную программу не требуется ее регистрация в какой-либо организации. Авторское право на компьютерную программу возникает автоматически при ее создании. Для оповещения о своих правах разработчик программы может, начиная с первого выпуска в свет программы, использовать знак охраны авторского права, состоящий из трех элементов:

\begin{itemize}
\item буквы С в окружности или круглых скобках \textcopyright;
\item наименования (имени) правообладателя;
\item года первого выпуска программы.       
\end{itemize}
Автору программы принадлежит исключительное право на воспроизведение и распространение программы любыми способами, а также на осуществление модификации программы.

\section{Понятии о информации, ее измерение}

\subsection{Информация и данные}

 Данные - это совокупность сведений, зафиксированных на определенном носителе в форме, пригодной для постоянного хранения, передачи и обработки. Преобразование и обработка данных позволяет получить информацию.

 Информация - это результат преобразования и анализа данных. Отличие информации от данных состоит в том, что данные - это фиксированные сведения о событиях и явлениях, которые хранятся на определенных носителях, а информация появляется в результате обработки данных при решении конкретных задач. Например, в базах данных хранятся различные данные, а по определенному запросу система управления базой данных выдает требуемую информацию.

 Применительно к компьютерной обработке данных под информацией понимают некоторую последовательность символических обозначений (букв, цифр, закодированных графических образов и звуков и т.п.), несущую смысловую нагрузку и представленную в понятном компьютеру виде. Каждый новый символ в такой последовательности символов увеличивает информационный объём сообщения.

\subsubsection{Операции с данными}

В ходе информационного процесса данные преобразуются из одного вида в другой. По мере развития НТП и общего усложнения связей в человеческом обществе трудозатраты на обработку данных неуклонно возрастают (постоянное усложнение условий управления производством и обществом + быстрые темпы появления и внедрения новых носителей/хранителей данных – увеличение объёма данных).

\begin{itemize}
\item Сбор – накопление данных с целью обеспечения достаточной полноты информации для принятия решения;

\item Формализаци – приведение данных, поступающих из разных источников, к одинаковой форме, чтобы сделать их сопоставимыми между собой, то есть повысить их уровень доступности;

\item Фильтрация – отсеивание «лишних» данных, в которых нет необходимости для принятия решений; при этом должен уменьшаться уровень «шума», а достоверность и адекватность данных должны возрастать;

\item Сортировка – упорядочение данных по заданному признаку с целью удобства использования; повышает доступность информации;

\item Группировка – объединение данных по заданному признаку с целью повышения удобства использования; повышает доступность информации;

\item Архиваци – организация хранения данных в удобной и легкодоступной форме; служит для снижения экономических затрат на хранение данных и повышает общую надежность информационного процесса в целом;

\item Защита – комплекс мер, направленных на предотвращение утраты, воспроизведение и модификации данных;

\item Транспортировка – прием и передача (доставка и поставка) данных между удаленными участниками информационного процесса; при этом источник данных в информатике принято называть сервером, а потребителя – клиентом;

\item Преобразование – перевод данных из одной формы в другую или из одной структуры в другую. Пример: изменение типа носителя; книги – бумага, электронная форма, микрофотоплёнка. Необходимость в многократном преобразовании данных возникает также при их транспортировке, особенно если она осуществляется средствами, не предназначенными для транспортировки данного вида данных.

\end{itemize}

\subsection{Свойства информации}

 При работе с информацией всегда имеются ее источник и потребитель (получатель). Для потребителя всегда очень важны свойства получаемой информации. Полезная информация уменьшает степень неопределенности у получателя и пополняет знания. Полезность информации относительна – кому-то полезна, а кому-то бесполезна. Данные становятся полезной информацией, если поступили своевременно, представляют интерес, новизну для решения практических задач. В противном случае данные бесполезны.

Можно привести немало разнообразных свойств информации. С точки зрения информатики наиболее важными представляются следующие свойства: адекватность, достоверность, полнота, актуальность и доступность, объективность информации.

\begin{itemize}
\item Адекватность информации – уровень соответствия создаваемого с помощью информации образа реальному объекту, процессу, явлению. Неадекватная информация может образовываться при создании новой информации на основе неполных или недостоверных данных. Неправильная информация – следствие предоставления неверных или искаженных сведений, ошибочной передачи, неправильно обработанных или ошибочных данных об объекте, событии или процессе. Однако и полные, и достоверные данные могут приводить к созданию неадекватной информации в случае применения к ним неадекватных методов. В реальной жизни человек вряд ли может рассчитывать на полную адекватность информации, так как всегда присутствует некоторая степень неопределенности. От степени адекватности информации реальному состоянию объекта или процесса зависит правильность принятия человеком решений. Адекватность информации может выражаться в трех формах: синтаксической, семантической и прагматической. Синтаксическая форма отражает формально-структурные характеристики и не затрагивает смысловое содержание информации. На синтаксическом уровне учитывается способ представления информации, скорость передачи информации и обработки, размеры кода представления информации. Рассматриваемую с этой синтаксической стороны информацию называют данными, так как при этом не имеет значения ее смысловая сторона.Семантическая форма отражает смысловое содержание информации. На этом уровне анализируются сведения, предоставляемые информацией, рассматриваются ее смысловые связи. Прагматический аспект отражает потребительскую сторону информации, ее соответствие цели управления, которая на основе этой информации реализуется. Он связан с ценностью, полезностью использования информации при выработке потребителем решения для достижения своей цели. С этой точки зрения анализируются потребительские свойства информации.

\item Достоверность информации – свойство отражать реально существующие объекты с необходимой точностью. Данные возникают в момент регистрации сигналов, но не все сигнал являются полезными – всегда присутствует какой-то уровень посторонних сигналов, в результате чего полезные данные сопровождаются определенным уровнем информационного шума. Информационный шум (информационный мусор) – данные и сведения, не несущие полезной информации, увеличивающие временные и прочие издержки пользователя при извлечении  и обработке информации. Если полезный сигнал зарегистрирован более четко, чем посторонние сигналы, достоверность информации может быть более высокой. При увеличении уровня шумов достоверность информации снижается. В этом случае для передачи того же количества информации требуется использовать либо больше данных, либо более сложные методы.Полнота информации характеризует качество информации и определяет достаточность данных для принятия решений или для создания новых данных на основе имеющихся. Неточная информация – недостаточные, неточные, неполные сведения об объекте, событии или процессе. Может использоваться для поиска решения задачи, по увеличивает вероятность неправильных выводов и поэтому требует уточнения, обновления данных.Чем полнее данные, тем шире диапазон методов, которые можно использовать, тем проще подобрать метод, вносящий минимум погрешностей в ход информационного процесса.

\item Актуальность информации – степень соответствия информации текущему моменту времени. Достоверная и адекватная, но устаревшая информация может приводить к ошибочным решениям. Необходимость поиска (или разработки) адекватного метода для работы с данными способна приводить к такой задержке в получении информации, что она становится неактуальной и ненужной.

\item Доступность информации – мера возможности получить ту или иную информацию. На степень доступности информации влияют одновременно как доступность данных, так и доступность адекватных методов для их интерпретации. Отсутствие доступа к данным или отсутствие адекватных методов обработки данных приводят к одинаковому результату: информация оказывается недоступной. Отсутствие адекватных методов для работы с данными во многих случаях приводит к применению неадекватных методов, в результате чего образуется неполная, неадекватная или недостоверная информация.

\item Объективность и субъективность информации. Понятие объективности информации является относительным. Более объективной принято считать ту информацию, в которую методы вносят меньший субъективный элемент. Так, в результате наблюдения фотоснимка природного объекта или явления образуется более объективная информация, чем в результате наблюдения рисунка того же объекта, выполненного человеком.

\end{itemize}

\subsection{Измерение информации}

Какое количество информации содержится, к примеру, в тексте романа 'Война и мир', в фресках Рафаэля или в генетическом коде человека? Ответа на эти вопросы наука не даёт и, по всей вероятности, даст не скоро. А возможно ли объективно измерить количество информации? Важнейшим результатом теории информации является вывод:

В определенных, весьма широких условиях можно пренебречь качественными особенностями информации, выразить её количество числом, а также сравнить количество информации, содержащейся в различных группах данных. В настоящее время получили распространение подходы к определению понятия 'количество информации', основанные на том, что информацию, содержащуюся в сообщении, можно нестрого трактовать в смысле её новизны или, иначе, уменьшения неопределённости наших знаний об объекте.
Так, американский инженер Р. Хартли (1928 г.) процесс получения информации рассматривает как выбор одного сообщения из конечного наперёд заданного множества из N равновероятных сообщений, а количество информации I, содержащееся в выбранном сообщении, определяет как двоичный логарифм N.
\begin{equation}
  I = \log_{2}N
\end{equation}
Допустим, нужно угадать одно число из набора чисел от единицы до ста. По формуле Хартли можно вычислить, какое количество информации для этого требуется: $I=\log_{2}100=6,644$. То есть сообщение о верно угаданном числе содержит количество информации, приблизительно равное 6,644 единиц информации.
Приведем другие примеры равновероятных сообщений:
\begin{enumerate}
\item при бросании монеты: 'выпала решка', 'выпал орел';
\item на странице книги: 'количество букв чётное', 'количество букв нечётное'.
\end{enumerate}

Определим теперь, являются ли равновероятными сообщения 'первой выйдет из дверей здания женщина' и 'первым выйдет из дверей здания мужчина'. Однозначно ответить на этот вопрос нельзя. Все зависит от того, о каком именно здании идет речь. Если это, например, станция метро, то вероятность выйти из дверей первым одинакова для мужчины и женщины, а если это военная казарма, то для мужчины эта вероятность значительно выше, чем для женщины. Для задач такого рода американский учёный Клод Шеннон предложил в 1948 г. другую формулу определения количества информации, учитывающую возможную неодинаковую вероятность сообщений в наборе.

\begin{equation}
  I = -\sum_{i=1}^{N}p_{i}\log_{2}p_{i}
\end{equation}

\begin{center}
  где, $p_{i}$ - вероятность того, что именно i-е сообщение выделено в наборе из N сообщений.
\end{center}

Легко заметить, что если вероятности $p_{1}, ... , p_{N}$ равны, то каждая из них равна $1/N$, и формула Шеннона превращается в формулу Хартли.

Помимо двух рассмотренных подходов к определению количества информации, существуют и другие. Важно помнить, что любые теоретические результаты применимы лишь к определённому кругу случаев, очерченному первоначальными допущениями.

В качестве единицы информации условились принять один бит (англ. bit — binary, digit — двоичная цифра). Бит в теории информации — количество информации, необходимое для различения двух равновероятных сообщений. А в вычислительной технике битом называют наименьшую 'порцию' памяти, необходимую для хранения одного из двух знаков '0' и '1', используемых для внутримашинного представления данных и команд.
Бит — слишком мелкая единица измерения. На практике чаще применяется более крупная единица — байт, равная восьми битам. Именно восемь битов требуется для того, чтобы закодировать любой из 256 символов алфавита клавиатуры компьютера (256 = $2^{8}$).

\begin{itemize}
  \item 1 слово = 2 байта = 16 бит
  \item 1  тетрада (нибл) = 1/2 байта = 4 бита
\end{itemize}

Широко используются также ещё более крупные производные единицы информации:
\begin{itemize}
  \item 1 Килобайт (Кбайт) = 1024 байт = $2^{10}$ байт,
  \item 1 Мегабайт (Мбайт) = 1024 Кбайт = $2^{20}$ байт,
  \item 1 Гигабайт (Гбайт) = 1024 Мбайт = $2^{30}$ байт.
\end{itemize}

В последнее время в связи с увеличением объёмов обрабатываемой информации входят в употребление такие производные единицы, как:
\begin{itemize}
  \item 1 Терабайт (Тбайт) = 1024 Гбайт = $2^{40}$ байт,
  \item 1 Петабайт (Пбайт) = 1024 Тбайт = $2^{50}$ байт,
  \item 1 Экабайт = $2^{64}$ байт.
\end{itemize}

За единицу информации можно было бы выбрать количество информации, необходимое для различения, например, десяти равновероятных сообщений. Это будет не двоичная (бит), а десятичная (дит) единица информации.

\section{Информационные ресурсы. Информационные системы и технологии}

\subsection{Информационные ресурсы}

Информационные ресурсы – это идеи человечества и указания по их реализации, накопленные в форме, позволяющей их воспроизводство. Это книги, статьи, патенты, диссертации, научно-исследовательская и опытно-конструкторская документация, технические переводы, данные о передовом производственном опыте и др. Информационные ресурсы (в отличие от всех других видов ресурсов — трудовых, энергетических, минеральных и т.д.) тем быстрее растут, чем больше их расходуют.

Система – любой объект, который одновременно рассматривается и как единое целое, и как объединенная в интересах достижения поставленной цели совокупность разнородных элементов.
Основные понятия:
\begin{enumerate}
\item Элемент системы – часть системы, имеющая определенное функциональное назначение. Сложные элементы, в свою очередь состоящие из более простых взаимосвязанных элементов, часто называют подсистемами.
\item  Организация системы – внутренняя упорядоченность, согласованность взаимодействия элементов системы, проявляющаяся, в частности, в ограничении разнообразия состояний элементов системы.
\item Структура системы – состав, порядок и принципы взаимодействия элементов системы, определяющая ее основные свойства. (например – иерархическая).
\item Архитектура системы – совокупность свойств системы, существенных для пользователя.
\item Целостность системы – принципиальная несводимость свойств системы к сумме свойств отдельных ее элементов и в то же время зависимость свойств каждого элемента от его места и функции внутри системы.
\end{enumerate}

\subsection{Процессы в информационной системе}
Информационная система – взаимосвязанная совокупность средств, методов и персонала, используемых для хранения, обработки и выдачи информации в интересах достижения поставленной цели.
  ИС определяется следующими свойствами:
  \begin{itemize}
    \item Любая ИС может быть подвергнута анализу, построена и управляема на основе общих принципов построения систем;
    \item ИС является динамичной и развивающейся;
    \item При построении ИС необходимо использовать системный подход;
    \item Выходной продукцией ИС является информация, на основе которой принимаются решения;
    \item ИС следует понимать как человеко-компьютерную систему обработки информации.
  \end{itemize}

  Структуру ИС составляет совокупность отдельных ее частей, называемых подсистемами. Подсистема – это часть системы, выделенная по какому-либо признаку.Подсистемы ИС называют обеспечивающими. Среди них выделяют:
  \begin{enumerate}
    \item Информационное обеспечение (informational support) – система классификации и кодирования информации, технологическая схема обработки данных, нормативно-справочная информация, система документооборота.
    \item Организационное обеспечение (organizational support) – совокупность мер и мероприятий, регламентирующих функционирование системы управления, взаимодействие работников с техническими средствами и между собой, описание системы, инструкции и регламенты обслуживающему персоналу.
    \item Техническое обеспечение (hardware) – комплекс используемых в системе управления технических средств, включающий ЭВМ и средства связи, а также документация на эти средства и технологические  процессы.
    \item Математическое обеспечение (mathematical support) – совокупность методов, правил, математических моделей и алгоритмов решения задач.
    \item Лингвистическое обеспечение (linguistic support) – совокупность терминов и искусственных языков, правил формализации естественного языка.
    \item Программное обеспечение (software) – совокупность программ систем обработки данных и документов, необходимых для эксплуатации этих программ. (Системное и прикладное ПО).
    \item Правовое обеспечение (legal support) – совокупность правовых норм, определяющих создание, юридический статус и функционирование системы, регламентирующих порядок получения, преобразования и использования информации.
  \end{enumerate}
  
  Информационная технология (ИТ) -
  \begin{itemize}
      \item система процедур преобразования информации с целью формирования, организации, обработки, распространения и использования информации;
      \item процесс, использующий совокупность средств и методов сбора, обработки и передачи данных (первичной информации) для получения информации нового качества о состоянии объекта, процесса или явления.
\end{itemize}
Этапы ИТ (виды инструментария):
\begin{enumerate}
  \item Ручная  (до 2-й половины 19 в.) - Перо, чернила, бумага. Почта, курьеры.
  \item Механическая (с кон. 19 в.). – Пишущая машинка, телефон, диктофон. Почта с более совершенными средствами доставки.
  \item Электрическая (40-60-е гг. 20в.) – Большие ЭВМ, электр. Пиш. Машинки, портативные диктофоны, ксероксы.
  \item  Электронная (с нач. 70-х гг. 20в.). – АСУ и информационно-поисковые системы (ИПС) на базе больших ЭВМ.
  \item Новая (с сер. 80-х гг. 20в.).  – Персональные ЭВМ, персонализация АСУ, системы искусственного интеллекта для поддержки принятия решений.
\end{enumerate}

Основу современных ИТ составляют:
\begin{itemize}
  \item Компьютерная обработка информации по заданным алгоритмам;
  \item Хранение больших объемов информации на машинных носителях;
  \item Передача информации на любое расстояние в ограниченное время.
\end{itemize}

\subsubsection{Сетевые технологии обработки данных}

---

\section{Представление данных в ЭВМ}

\subsection{Кодирование информации}
 В процессе преобразования информации из одной формы представления (знаковой системы) в другую осуществляется кодирование. Средством кодирования служит таблица соответствия, которая устанавливает взаимно однозначное соответствие между знаками или группами знаков двух различных знаковых систем.В процессе обмена информацией часто приходится производить операции кодирования и декодирования информации. При вводе знака алфавита в компьютер путем нажатия соответствующей клавиши на клавиатуре выполняется его кодирование, т. е. преобразование в компьютерный код. При выводе знака на экран монитора или принтер происходит обратный процесс — декодирование, когда из компьютерного кода знак преобразуется в графическое изображение.
     
 \subsubsection{Кодирование изображений и звука}
 Информация, в том числе графическая и звуковая, может быть представлена в аналоговой или дискретной форме. При аналоговом представлении физическая величина принимает бесконечное множество значений, причем ее значения изменяются непрерывно. При дискретном представлении физическая величина принимает конечное множество значений, причем ее величина изменяется скачкообразно.

 Примером аналогового представления графической информации может служить, скажем, живописное полотно, цвет которого изменяется непрерывно, а дискретного — изображение, напечатанное с помощью струйного принтера и состоящее из отдельных точек разного цвета.Примером аналогового хранения звуковой информации является виниловая пластинка (звуковая дорожка изменяет свою форму непрерывно), а дискретного — аудиокомпакт-диск (звуковая дорожка которого содержит участки с различной отражающей способностью).
 
 Графическая и звуковая информация из аналоговой формы в дискретную преобразуется путем дискретизации, т. е. разбиения непрерывного графического изображения и непрерывного (аналогового) звукового сигнала на отдельные элементы. В процессе дискретизации производится кодирование, т. е. присвоение каждому элементу конкретного значения в форме кода.
 
 Дискретизация — это преобразование непрерывных изображений и звука в набор дискретных значений, каждому из которых присваивается значение его кода.
 
 Кодирование информации в живых организмах. Генетическая информация определяет строение и развитие живых организмов и передается по наследству. Хранится генетическая информация в клетках организмов в структуре молекул ДНК (дезоксирибонуклеиновой кислоты). Молекулы ДНК состоят из четырех различных составляющих (нуклеотидов), которые образуют генетический алфавит. Молекула ДНК человека включает в себя около трех миллиардов пар нуклеотидов, и в ней закодирована вся информация об организме человека: его внешность, здоровье или предрасположенность к болезням, способности и т. д.

\subsubsection{Двоичная система счисления}
Система счисления — это знаковая система, в которой числа записываются по определенным правилам с помощью цифр — символов некоторого алфавита. Например, в десятичной системе для записи числа существует десять всем хорошо известных цифр: О, 1, 2 и т. д.

Все системы счисления делятся на позиционные и непозиционные. В позиционных системах счисления значение цифры зависит от ее положения в записи числа, а в непозиционных — не зависит. Позиция цифры в числе называется разрядом. Разряд числа возрастает справа налево, от младших разрядов к старшим. В непозиционных системах вес цифры (т.е. тот вклад, который она вносит в значение числа) не зависит от ее позиции в записи числа. Так, в римской системе счисления в числе ХХХII (тридцать два) вес цифры Х в любой позиции равен просто десяти.В позиционных системах счисления вес каждой цифры изменяется в зависимости от ее положения (позиции) в последовательности цифр, изображающих число. Например, в числе 757,7 первая семерка означает 7 сотен, вторая – 7 единиц, а третья – 7 десятых долей единицы. Сама же запись числа 757,7 означает сокращенную запись выражения:
\begin{equation}
  700 + 50 + 7 + 0,7 = 7*10_{2} + 5*10_{1} + 7*10_{0} + 7*10_{-1} = 757,7
\end{equation}

Любая позиционная система счисления характеризуется своим основанием. Основание позиционной системы счисления — это количество различных знаков или символов, используемых для изображения цифр в данной системе. За основание системы можно принять любое натуральное число — два, три, четыре и т.д. Следовательно, возможно бесчисленное множество позиционных систем: двоичная, троичная, четверичная и т.д. Запись чисел в каждой из систем счисления с основанием q означает сокращенную запись выражения:
\begin{equation}
  a_{n-1} * q^{n-1} + a_{n-2}* q^{n-2} + ... + a_{1}* q^{1} + a_{0}*q^{0} + a_{-1}*q^{-1} + ... + a_{-m} * q^{-m}
\end{equation}
\begin{center}
  где $a_{i}$ – цифры системы счисления; $n$ и $m$ – число целых и дробных разрядов, соответственно.
\end{center}
Каждая позиционная система использует определенный алфавит цифр и основание. В позиционных системах счисления основание системы равно количеству цифр (знаков в ее алфавите) и определяет, во сколько раз различаются значения цифр соседних разрядов числа.
Кроме десятичной широко используются системы с основанием, являющимся целой степенью числа 2, а именно:
\begin{itemize}
  \item двоичная (используются цифры 0, 1);
  \item восьмеричная (используются цифры 0, 1, ..., 7);
  \item шестнадцатеричная (для первых целых чисел от нуля до девяти используются цифры 0, 1, ..., 9, а для следующих чисел — от десяти до пятнадцати – в качестве цифр используются символы A, B, C, D, E, F).
\end{itemize}
% В таблице 1 указаны первые шестнадцать чисел этих систем счисления.
\begin{table}
  \label{nums}
      \caption{Запись первых шестнадцати целых чисел в разных системах счисления}
      \begin{center}
      \begin{tabular}{|c|c|c|c|}
        \hline
        двоичная & восьмеричная & десятичная & шестнадцатеричная \\
        \hline
        00000 & 0 & 0 & 0 \\
        00001 & 1 & 1 & 1 \\
        00010 & 2 & 2 & 2 \\
        00011 & 3 & 3 & 3 \\
        00100 & 4 & 4 & 4 \\
        00101 & 5 & 5 & 5 \\
        00110 & 6 & 6 & 6 \\
        00111 & 7 & 7 & 7 \\
        01000 & 10 & 8 & 8 \\
        01001 & 11 & 9 & 9 \\
        01010 & 12 & 10 & A \\
        01011 & 13 & 11 & B \\
        01100 & 14 & 12 & C \\
        01101 & 15 & 13 & D \\
        01110 & 16 & 14 & E \\
        01111 & 17 & 15 & F \\
        10000 & 20 & 16 & 10 \\
        \hline
      \end{tabular}
      \end{center}
    \end{table}

Из всех систем счисления особенно проста и поэтому интересна для технической реализации в компьютерах двоичная система счисления.

Рассмотрим в качестве примера десятичное число 555. Цифра 5 встречается трижды, причем самая правая обозначает пять единиц, вторая справа — пять десятков и, наконец, третья — пять сотен. Число 555 записано в привычной для нас свернутой форме. Мы настолько привыкли к такой форме записи, что уже не замечаем, как в уме умножаем цифры числа на различные степени числа 10. В развернутой форме запись числа 555 в десятичной системе выглядит следующим образом:
$$555_{10} = 5 * 10^{2} + 5 * 10^{1} + 5 * 10^{0}$$
Как видно из примера, число в позиционных системах счисления записывается в виде суммы степеней основания (в данном случае 10), коэффициентами при этом являются цифры данного числа. В двоичной системе основание равно 2, а алфавит состоит из двух цифр (0 и 1). В развернутой форме двоичные числа записываются в виде суммы степеней основания 2 с коэффициентами, в качестве которых выступают цифры 0 или 1. Например, развернутая запись двоичного числа $101_{2}$ в десятичном представлении будет иметь вид:
$$1 * 2^{2} + 0 * 2^{1} + 1 * 2^{0}$$

\subsection{Выполнение арифметических операций в двоичной системе счисления}



\end{document}                                                                               
