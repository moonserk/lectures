\documentclass[a4paper]{article}
\usepackage[T1,T2A]{fontenc}
\usepackage[utf8]{inputenc}
\usepackage[english,russian]{babel}
%\usepackage{amsmath}
%\usepackage{amsfonts}
%\usepackage{amssymb}
%\usepackage{makeidx}

\usepackage[english,russian]{babel}
\begin{document}

\section{Информатика, ее составляющие и роль в современном мире}

\subsection{Информатика, ее место в системе наук}

Термин информатика возник в 60-х гг. во Франции для названия области, занимающейся автоматизированной обработкой информации с помощью электронных вычислительных машин. Французский термин informatigue (информатика)образован путем слияния слов information (информация) и automatigue (автоматика) и означает "информационная автоматика или автоматизированная переработка информации". В англоязычных странах этому термину соответствует синоним computer science (наука о компьютерной технике).

Выделение информатики как самостоятельной области человеческой деятельности в первую очередь связано с развитием компьютерной техники. Причем основная заслуга в этом принадлежит микропроцессорной технике, появление которой в середине 70-х гг. послужило началом второй электронной революции. С этого времени элементной базой вычислительной машины становятся интегральные схемы и микропроцессоры, а область, связанная с созданием и использованием компьютеров, получила мощный импульс в своем развитии. Термин "информатика'' приобретает новое дыхание и используется не только для отображения достижений компьютерной техники, но и связывается с процессами передачи и обработки информации.

В нашей стране подобная трактовка термина 'информатика' утвердилась с момента принятия решения в 1983 г. на сессии годичного собрания Академии наук СССР об организации нового отделения информатики, вычислительной техники и автоматизации. Информатика трактовалась как 'комплексная научная и инженерная дисциплина, изучающая все аспекты разработки, проектирования, создания, оценки, функционирования основанных на ЭВМ систем переработки информации, их применения и воздействия на различные области социальной практики'. Информатика в таком понимании нацелена на разработку общих методологических принципов построения информационных моделей. Поэтому методы информатики применимы всюду, где существует возможность описания объекта, явления, процесса и т.п. с помощью информационных моделей.

Существует множество определений информатики, что связано с многогранностью ее функций, возможностей, средств и методов. Один из вариантов которые предлагает википедия: Информáтика (фр. Informatique) — наука о методах и процессах сбора, хранения, обработки, передачи, анализа и оценки информации с нием компьютерных технологий, обеспечивающих возможность её использования для принятия решений. Информатика включает дисциплины, относящиеся к обработке информации в вычислительных машинах и вычислительных сетях: как абстрактные, вроде анализа алгоритмов, так и конкретные, например, разработка языков программирования и протоколов передачи данных.

Темами исследований в информатике как науки являются вопросы: что можно, а что нельзя реализовать в программах и базах данных (теория вычислимости и искусственный интеллект), каким образом можно решать специфические вычислительные и информационные задачи с максимальной эффективностью (теория сложности вычислений), в каком виде следует хранить и восстанавливать информацию специфического вида (структуры и базы данных), как программы и люди должны взаимодействовать друг с другом (пользовательский интерфейс и языки программирования и представление знаний) и т. п.

Информатика делится на ряд разделов:

Теоретическая информатика

Огромное поле исследований теоретической информатики включает как классическую теорию алгоритмов, так и широкий спектр тем, связанных с более абстрактными логическими и математическими аспектами вычислений. Теоретическая информатика занимается теориями формальных языков, автоматов, алгоритмов, вычислимости и вычислительной сложности, а также вычислительной теорией графов, криптологией, логикой (включая логику высказываний и логику предикатов), формальной семантикой и закладывает теоретические основы для разработки компиляторов языков программирования.

Прикладная информатика

Прикладная информатика направлена на применение понятий и результатов теоретической информатики к решению конкретных задач в конкретных прикладных областях.

\subsection{Связь информатики с другими науками}

---

\subsection{Экономические, социальные и правовые аспекты информационных технологий}

---

\section{Понятии о информации, ее измерение}

\subsection{Информация и данные}

 Данные - это совокупность сведений, зафиксированных на определенном носителе в форме, пригодной для постоянного хранения, передачи и обработки. Преобразование и обработка данных позволяет получить информацию.

Информация - это результат преобразования и анализа данных. Отличие информации от данных состоит в том, что данные - это фиксированные сведения о событиях и явлениях, которые хранятся на определенных носителях, а информация появляется в результате обработки данных при решении конкретных задач. Например, в базах данных хранятся различные данные, а по определенному запросу система управления базой данных выдает требуемую информацию.

 Операции с данными

В ходе информационного процесса данные преобразуются из одного вида в другой. По мере развития НТП и общего усложнения связей в человеческом обществе трудозатраты на обработку данных неуклонно возрастают (постоянное усложнение условий управления производством и обществом + быстрые темпы появления и внедрения новых носителей/хранителей данных – увеличение объёма данных).


\begin{enumerate}
\item[Сбор] – накопление данных с целью обеспечения достаточной полноты информации для принятия решения;

\item[Формализация] – приведение данных, поступающих из разных источников, к одинаковой форме, чтобы сделать их сопоставимыми между собой, то есть повысить их уровень доступности;

\item[Фильтрация] – отсеивание «лишних» данных, в которых нет необходимости для принятия решений; при этом должен уменьшаться уровень «шума», а достоверность и адекватность данных должны возрастать;

\item[Сортировка] – упорядочение данных по заданному признаку с целью удобства использования; повышает доступность информации;

\item[Группировка] – объединение данных по заданному признаку с целью повышения удобства использования; повышает доступность информации;

\item[Архиваци] – организация хранения данных в удобной и легкодоступной форме; служит для снижения экономических затрат на хранение данных и повышает общую надежность информационного процесса в целом;

\item[Защита] – комплекс мер, направленных на предотвращение утраты, воспроизведение и модификации данных;

\item[Транспортировка] – прием и передача (доставка и поставка) данных между удаленными участниками информационного процесса; при этом источник данных в информатике принято называть сервером, а потребителя – клиентом;

\item[Преобразование] – перевод данных из одной формы в другую или из одной структуры в другую. Пример: изменение типа носителя; книги – бумага, электронная форма, микрофотоплёнка. Необходимость в многократном преобразовании данных возникает также при их транспортировке, особенно если она осуществляется средствами, не предназначенными для транспортировки данного вида данных.

\end{enumerate}

\subsection{Свойства информации}

 При работе с информацией всегда имеются ее источник и потребитель (получатель). Для потребителя всегда очень важны свойства получаемой информации. Полезная информация уменьшает степень неопределенности у получателя и пополняет знания. Полезность информации относительна – кому-то полезна, а кому-то бесполезна. Данные становятся полезной информацией, если поступили своевременно, представляют интерес, новизну для решения практических задач. В противном случае данные бесполезны.

Можно привести немало разнообразных свойств информации. С точки зрения информатики наиболее важными представляются следующие свойства: адекватность, достоверность, полнота, актуальность и доступность, объективность информации.

\begin{enumerate}
\item Адекватность информации – уровень соответствия создаваемого с помощью информации образа реальному объекту, процессу, явлению. Неадекватная информация может образовываться при создании новой информации на основе неполных или недостоверных данных. Неправильная информация – следствие предоставления неверных или искаженных сведений, ошибочной передачи, неправильно обработанных или ошибочных данных об объекте, событии или процессе. Однако и полные, и достоверные данные могут приводить к созданию неадекватной информации в случае применения к ним неадекватных методов. В реальной жизни человек вряд ли может рассчитывать на полную адекватность информации, так как всегда присутствует некоторая степень неопределенности. От степени адекватности информации реальному состоянию объекта или процесса зависит правильность принятия человеком решений. Адекватность информации может выражаться в трех формах: синтаксической, семантической и прагматической. Синтаксическая форма отражает формально-структурные характеристики и не затрагивает смысловое содержание информации. На синтаксическом уровне учитывается способ представления информации, скорость передачи информации и обработки, размеры кода представления информации. Рассматриваемую с этой синтаксической стороны информацию называют данными, так как при этом не имеет значения ее смысловая сторона.

Семантическая форма отражает смысловое содержание информации. На этом уровне анализируются сведения, предоставляемые информацией, рассматриваются ее смысловые связи. Прагматический аспект отражает потребительскую сторону информации, ее соответствие цели управления, которая на основе этой информации реализуется. Он связан с ценностью, полезностью использования информации при выработке потребителем решения для достижения своей цели. С этой точки зрения анализируются потребительские свойства информации.

\item Достоверность информации – свойство отражать реально существующие объекты с необходимой точностью. Данные возникают в момент регистрации сигналов, но не все сигнал являются полезными – всегда присутствует какой-то уровень посторонних сигналов, в результате чего полезные данные сопровождаются определенным уровнем информационного шума. Информационный шум (информационный мусор) – данные и сведения, не несущие полезной информации, увеличивающие временные и прочие издержки пользователя при извлечении  и обработке информации. Если полезный сигнал зарегистрирован более четко, чем посторонние сигналы, достоверность информации может быть более высокой. При увеличении уровня шумов достоверность информации снижается. В этом случае для передачи того же количества информации требуется использовать либо больше данных, либо более сложные методы.

Полнота информации характеризует качество информации и определяет достаточность данных для принятия решений или для создания новых данных на основе имеющихся. Неточная информация – недостаточные, неточные, неполные сведения об объекте, событии или процессе. Может использоваться для поиска решения задачи, по увеличивает вероятность неправильных выводов и поэтому требует уточнения, обновления данных.

Чем полнее данные, тем шире диапазон методов, которые можно использовать, тем проще подобрать метод, вносящий минимум погрешностей в ход информационного процесса.

\item Актуальность информации – степень соответствия информации текущему моменту времени. Достоверная и адекватная, но устаревшая информация может приводить к ошибочным решениям. Необходимость поиска (или разработки) адекватного метода для работы с данными способна приводить к такой задержке в получении информации, что она становится неактуальной и ненужной.

\item Доступность информации – мера возможности получить ту или иную информацию. На степень доступности информации влияют одновременно как доступность данных, так и доступность адекватных методов для их интерпретации. Отсутствие доступа к данным или отсутствие адекватных методов обработки данных приводят к одинаковому результату: информация оказывается недоступной. Отсутствие адекватных методов для работы с данными во многих случаях приводит к применению неадекватных методов, в результате чего образуется неполная, неадекватная или недостоверная информация.

Объективность и субъективность информации. Понятие объективности информации является относительным. Более объективной принято считать ту информацию, в которую методы вносят меньший субъективный элемент. Так, в результате наблюдения фотоснимка природного объекта или явления образуется более объективная информация, чем в результате наблюдения рисунка того же объекта, выполненного человеком.

\end{enumerate}

\section{Информационные ресурсы. Информационные системы и технологии}

\section{Представление данных в ЭВМ}

\end{document}       
