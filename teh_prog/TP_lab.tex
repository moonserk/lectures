\documentclass[a4paper]{article}
\usepackage[T1,T2A]{fontenc}
\usepackage[utf8]{inputenc}
\usepackage[english,russian]{babel}
\usepackage{booktabs}
\usepackage{color,colortbl}
%\usepackage{amsmath}
%\usepackage{amsfonts}
%\usepackage{amssymb}
%\usepackage{makeidx}
\usepackage{listings}
\usepackage{graphicx}
\usepackage{rotating}
\definecolor{green}{RGB}{45,140,31}
\definecolor{darkishgreen}{RGB}{39,203,22}
\definecolor{LightCyan}{rgb}{0.88,1,1}
\definecolor{Gray}{gray}{0.9}
\definecolor{lightRed}{RGB}{230,170,150}
\definecolor{modRed}{RGB}{230,82,90}
\definecolor{strongRed}{RGB}{230,6,6}

\lstset{ %
  language=python,                % Язык программирования
  numbers=left,                   % С какой стороны нумеровать
  extendedchars=\true,
  %numberstyle=tinycolor{gray},     % Стиль который будет использоваться для нумерации строк
  %stepnumber=2,                   % Шаг между линиями. Если 1, то будет пронумерована каждая строка
 % numbersep=5pt,
 % backgroundcolor=color{white},      % Цвет подложки. Вы должны добавить пакет color - usepackage{color}
  showspaces=false,
  showstringspaces=false,
  showtabs=false,
  %frame=single,                    % Добавить рамку
  %rulecolor=color{black},
  tabsize=4,                       % Tab - 2 пробела
  breaklines=true,                 % Автоматический перенос строк
  breakatwhitespace=true,          % Переносить строки по словам
  title=lstname,                   % Показать название подгружаемого файла
  keywordstyle=\color{green},          % Стиль ключевых слов
  %commentstyle=color{dkgreen},       % Стиль комментариев
  %stringstyle=color{mauve}          % Стиль литералов
}

\usepackage[english,russian]{babel}

\begin{document}

\begin{titlepage}
  \newpage
  \begin{center}
    {\bfseries Сарапульский политехнический институт (филиал)
      федерального государственного бюджетного образовательного
      учреждение высшего образования
      «Ижевский государственный технический университет имени М.Т. Калашникова» \linebreak
      (СПИ (филиал) ФГБОУ ВО «ИжГТУ имени \linebreak
      М.Т. Калашникова»)
    }
  \end{center}
    \begin{center}
      % \textsc{\textbf{}}
      \topskip=0pt
      \vspace*{\fill}
      \large \textbf{Технология программирования} \linebreak
      методические указания к выполнению лаборатрных работ \linebreak
      для студентов направления 20.03.01 \linebreak
      «Техносферная безопасность» \linebreak
      для всех форм обучения 
      \vspace*{\fill}
    \end{center}
    \begin{center}
      Составитель:
      \hspace*{\fill} старший преподаватель \linebreak
      \hspace*{\fill} Романцов Г. Д.
    \end{center}
     \begin{center}
       \vspace*{\fill}
       Сарапул\linebreak 2020
    \end{center}
  \end{titlepage}

  \tableofcontents

\newpage
\section{Лабораторная работа №2.\newline Разработка интерфейса приложения}

Цель работы: Научиться создавать приложения с графическим пользовательским интерфейсом.

\subsection{Теоретические основы}

\textbf{GUI (Graphical User Interface) или ГИП (графический интерфейс пользователя)} --- это одна из разновидностей пользовательских интерфейсов, элементы которого выполнены в виде графических изображений. То есть все основные объекты, присутствующие в этом интерфейсе — иконки, функциональные кнопки, объекты меню и т.д. — выполнены в виде изображений. Если сравнить GUI с обычной командной строкой, то в первом варианте перед пользователем открывается полный доступ к абсолютно всем элементам, который он видит на дисплее. Реализовать этот доступ можно с использованием разных устройств ввода: оптической мыши, трекбола, клавиатуры, джойстика и пр. Обычно в GUI каждый графический объект передает смысл функции с помощью понятного образа, чтобы пользователю было проще разобраться с определенным программным обеспечением и легче взаимодействовать с ОС в целом. Но важно понимать, что GUI — это лишь составная часть графического интерфейса. Функционирует он на уровне визуализации данных и таким же образом взаимодействует с пользователем. Одно из ключевых требований к хорошему GUI — реализация концепции DWIM (Do What I Mean или дословно «делай то, что я имею в виду»). То есть система должна функционировать предсказуемо, чтобы пользователь интуитивно понимал, что произойдет после его определенного действия (ввода команды).

\textbf{Конструктор графического пользовательского интерфейса» (или «GUI-конструктор»)}, также известный как «GUI-редактор», является инструментарием разработки программного обеспечения, который упрощает создание графического интерфейса пользователя (GUI), позволяя разработчику упорядочить элементы интерфейса (часто называемые виджетами) используя редактор drag-and-drop WYSIWYG. Без GUI-конструктора графический интерфейс пользователя должен быть создан вручную, указывая параметры каждого элемента интерфейса в исходном коде без визуальной обратной связи до запуска программы. Пользовательские интерфейсы обычно программируются с помощью событийно-ориентированной архитектуры, поэтому GUI-конструкторы также упрощают создание кода, управляемого событиями. Этот вспомогательный код соединяет элементы интерфейса с исходящими и входящими событиями, которые запускают функции, обеспечивающие логику работы приложения.

\textbf{Веб-интерфейс} --- веб-страница или совокупность веб-страниц, предоставляющая пользовательский интерфейс для взаимодействия с сервисом или устройством посредством протокола HTTP и веб-браузера. Веб-интерфейсы получили широкое распространение в связи с ростом популярности всемирной паутины и соответственно — повсеместного распространения веб-браузеров. Классическим и наиболее популярным методом создания веб-интерфейсов является использование HTML с применением CSS и JavaScript'a. Однако различная реализация HTML, CSS, DOM и других спецификаций в браузерах вызывает проблемы при разработке веб-приложений и их последующей поддержке. Кроме того, возможность пользователя настраивать многие параметры браузера (например, размер шрифта, цвета, отключение поддержки сценариев) может препятствовать корректной работе интерфейс

Существует множество бибилиотек для разных платформ и языков программирования. В данном примере будет использоватся кросс-платформенная событийно-ориентированная графическая библиотека на основе средств Tk входящая в стандартную библиотеку Python --- Tkinter. С помощью этой библиотеки можно легко писать простые GUI приложения.
Библиотека предназначена для организации диалогов в программе с помощью оконного графического интерфейса (GUI). В составе библиотеки присутствуют общие графические компоненты:

\begin{itemize}
\item Toplevel/Tk - Окно верхнего уровня (корневой виджет).
\item Frame - Рамка. Содержит в себе другие визуальные компоненты, используется для группировки виджетов.
\item Label - Этикетка. Показывает некоторый текст или графическое изображение.
\item Entry - Однострочное поле ввода текста.
\item Text - Форматируемое поле ввода текста. Позволяет показывать, редактировать и форматировать текст с использованием различных стилей, а также внедрять в текст рисунки и окна.
\item Canvas - Холст. Может использоваться для вывода графических примитивов, таких как прямоугольники, эллипсы, линии, а также текст, изображения и окна.
\item Button - Кнопка. Простая кнопка для выполнения команды и других действий.
\item Radiobutton - Переключатель. Представляет одно из альтернативных значений некоторой переменной. Обычно действует в группе. Когда пользователь выбирает какую-либо опцию, с ранее выбранного в этой же группе элемента выбор снимается.
\item Checkbutton - Флажок. Похож на Radiobutton, но имеет возможность множественного выбора, предоставляя отдельную переменную на каждый экземпляр виджета.
\item Scale - Шкала с ползунком. Позволяет задать числовое значение путём перемещения движка.
\item Listbox - Список. Показывает список, из которого пользователь может выделить один или несколько элементов.
\item Scrollbar - Полоса прокрутки. Может использоваться вместе с некоторыми другими компонентами для их прокрутки.
\item Menu - Меню. Служит для организации всплывающих (popup) и ниспадающих (pulldown) меню.
\item Menubutton - Кнопка-меню. Используется для организации pulldown-меню.
\item Message - Сообщение. Аналогично Label, но позволяет заворачивать длинные строки и легко меняет свой размер.
\end{itemize}

Для примера разработем простое приложения для создания списка дел:

\begin{lstlisting}[label=todo, caption=Список дел]
# Подключение библиотек
import tkinter as tk
from tkinter import ttk
from tkinter import messagebox

#Создание основного фрейма
root = tk.Tk()
root.title('Список дел')
root.geometry("400x400")

#Инициализация массива для добавления заданий
tasks = []

#Функция добавления задания
def addTask():
    word = input_entry.get()
    if len(word)==0:
        messagebox.showinfo('Отсутствуют данные', 'Введите задачу')
    else:
        tasks.append(word)
        listUpdate()
        input_entry.delete(0,'end')

#Функция удаления задания
def delOne():
    try:
        val = list_box.get(list_box.curselection())
        if val in tasks:
            tasks.remove(val)
            listUpdate()
    except:
            messagebox.showinfo('Невозможно удалить', 'Не выбрана задача')

#Фукция обновления листа с заданиями
def listUpdate():
    clearList()
    for i in tasks:
        list_box.insert('end', i)

#Функция очистки списка
def clearList():
    list_box.delete(0,'end')

#Функция выхода из приложения
def bye():
    root.destroy()

#Инициализация UI компонентов
title = ttk.Label(root, text = 'Список дел')
input_label = ttk.Label(root, text='Введите задачу: ')
input_entry = ttk.Entry(root, width=20)
list_box = tk.Listbox(root, height=15, selectmode='SINGLE')
add_btn = ttk.Button(root, text='Добавить', width=20, command=addTask)
del_btn = ttk.Button(root, text='Удалить', width=20, command=delOne)
exit_btn = ttk.Button(root, text='Выход', width=20, command=bye)

input_label.place(x=20, y=40)
input_entry.place(x=20, y=60)
add_btn.place(x=20, y=90)
del_btn.place(x=20, y=120)
exit_btn.place(x=20, y =305)
title.place(x=20, y=10)
list_box.place(x=220, y=60)

#Главный цикл обработки событий
root.mainloop()
\end{lstlisting}

\newpage
\subsection{Задание на лабораторную работу}

Для выполнения работы необходимо:
\begin{enumerate}
  \item Изучить теоретический материал
  \item Выбрать задание в соответствии с вариатом
  \item Написать программу с GUI
    \begin{itemize}
      \item Можно использовать любую библиотеку
      \item Можно использовать любой язык программирования (кроме 1С)
      \item Можно использовать GUI - конструкторы
      \item Можно сделать веб-интерфейс
    \end{itemize}
  \item Опубликовать код на github и предоставить ссылку в отчете
  \item Оформить отчет по лабораторной работе
\end{enumerate}

\subsection{Содержание отчета}
\begin{enumerate}
  \item Титульный Лист
  \item Цель работы
  \item Краткие теоретические сведения по теме лабораторной работы
  \item Выплоненое задание
  \item Ссылка на github
  \item Краткий вывод о проделанной работе
\end{enumerate}

\begin{thebibliography}{3}
  \bibitem{python}Сузи, Р. А. Язык программирования Python : учебное пособие / Р. А. Сузи. — 3-е изд. — Москва : Интернет-Университет Информационных Технологий (ИНТУИТ), Ай Пи Ар Медиа, 2020. — 350 c. — ISBN 978-5-4497-0705-5. — Текст : электронный // Электронно-библиотечная система IPR BOOKS : [сайт]. — URL: http://www.iprbookshop.ru/97589.html
  \bibitem{tkinter} Документация стандартной библиотеки Python : tkinter — Python interface to Tcl/Tk [cайт]. --- URL: https://docs.python.org/3/library/tkinter.html
\end{thebibliography}


\end{document}
