\documentclass[a4paper]{article}
\usepackage[T1,T2A]{fontenc}
\usepackage[utf8]{inputenc}
\usepackage[english,russian]{babel}
\usepackage{booktabs}
\usepackage{color,colortbl}
%\usepackage{amsmath}
%\usepackage{amsfonts}
%\usepackage{amssymb}
%\usepackage{makeidx}
\usepackage{listings}
\usepackage{graphicx}
\usepackage{rotating}
\definecolor{green}{RGB}{45,140,31}
\definecolor{darkishgreen}{RGB}{39,203,22}
\definecolor{LightCyan}{rgb}{0.88,1,1}
\definecolor{Gray}{gray}{0.9}
\definecolor{lightRed}{RGB}{230,170,150}
\definecolor{modRed}{RGB}{230,82,90}
\definecolor{strongRed}{RGB}{230,6,6}

\lstset{ %
  language=python,                % Язык программирования
  numbers=left,                   % С какой стороны нумеровать
  extendedchars=\true,
  %numberstyle=tinycolor{gray},     % Стиль который будет использоваться для нумерации строк
  %stepnumber=2,                   % Шаг между линиями. Если 1, то будет пронумерована каждая строка
 % numbersep=5pt,
 % backgroundcolor=color{white},      % Цвет подложки. Вы должны добавить пакет color - usepackage{color}
  showspaces=false,
  showstringspaces=false,
  showtabs=false,
  %frame=single,                    % Добавить рамку
  %rulecolor=color{black},
  tabsize=4,                       % Tab - 2 пробела
  breaklines=true,                 % Автоматический перенос строк
  breakatwhitespace=true,          % Переносить строки по словам
  title=lstname,                   % Показать название подгружаемого файла
  keywordstyle=\color{green},          % Стиль ключевых слов
  %commentstyle=color{dkgreen},       % Стиль комментариев
  %stringstyle=color{mauve}          % Стиль литералов
}

\usepackage[english,russian]{babel}

\begin{document}

\newpage
\section{Лабораторная работа №3.\newline Реализация диалоговых элементов в графическом пользовательском интерфейсе}

Цель работы: Научиться реализации диалоговых окон в приложениях.

\subsection{Теоретические основы}

\textbf{Диалоговым окном} называют специальный элемент интерфейса, предназначеный для вывода информации или получения информации от пользователя. Диалоговое окно обеспечивает двухсторонне взаимодействие ``Компьютер -> Пользователь''. Простейшим типом диалогового окна является окно сообщения, которое выводит сообщение и требует от пользователя подтвердить, что сообщение прочитано. Для этого обычно необходимо нажать кнопку OK. Окно сообщения предназначено для информирования пользователя о завершении выполнявшегося действия, вывода сообщения об ошибке и тому подобных случаев, не требующих от пользователя какого---либо выбора.

Диалоговые окна подразделяются на модальные и немодальные, в зависимости от того, блокируют ли они возможность взаимодействия пользователя с приложением (или системой в целом) до тех пор, пока не получат от него ответ. \textbf{Немодальные} диалоговые окна используются в случаях, когда выводимая информация в окне не является существенной для работы программы. Окно может оставатся открытым, в то время как работа пользователя с программой продолжается. азновидностью немодального окна является панель инструментов или окно-«палитра», если она отсоединена или может быть отсоединена от главного окна приложения, так как элементы управления, расположенные на ней, могут использоваться параллельно с работой приложения. \textbf{Модальным} называют окно которое блакирует работу программы, до тех пор пока пользователь не закроет окно. Например, модальными являются диалоговые окна настроек приложения — так как проще реализовать режим, когда все сделанные изменения настроек применяются или отменяются одномоментно, и с момента, когда пользователь решил изменить настройки приложения и открыл диалог настроек, и до момента, когда он новые настройки вводит в силу или отказывается от них, приложение ожидает решения пользователя.

Ранее разработанное нами приложения для списка дел не имеет диалоговых окон, все сообщения выводятся в консоль. Пользователь не получает никаких уведомлений, предупреждений и т. д. Как правило библиотеки для создания графического интерфейса пользователя так же предоставляют и элементы для создания диалоговых окон. Библиотека tkinker не исключение, она предоставляет пакет для работы с диалоговыми окнами ---
\textbf{messagebox}.

Что бы показать минималистичное диалоговое окно messagebox, использует метод showinfo где параметрами являются заголовок окна и текст. как, например, в листинге~\ref{messagebox}:

\newpage
\begin{lstlisting}[label=messagebox,caption=Добавление новой записи]
  messagebox.showinfo('Отсутствуют данные', 'Не выбрана задача')
\end{lstlisting}

Вообще messagebox имеет несколько видов диалоговых окон:

\begin{itemize}
        \item messagebox.showerror(``Ошибка'', ``Сообщение об ошибке'')
        \item messagebox.showwarning(``Предупреждение'', ``Текст предупреждения'')
        \item messagebox.showinfo(``Информация'', ``Информационное сообщение'')
\end{itemize}

Для отображения диалогового окна с вопросом испльзуется метод askyesno который возвращает булево значение в зависимости от выбора пользователя, а параметрами метода являются так же заголовок и текст. Пример в листинге~\ref{askyesno}

\begin{lstlisting}[label=askyesno,caption=Добавление новой записи]
    answer = messagebox.askyesno(title="Выход", message="Вы действительно хотите выйти?")
    if answer:
        root.destroy()
\end{lstlisting}

После добавления диалоговых окон в программу графический интерфейс пользователя стал более информативным.

\lstinputlisting[caption='Список дел с диалоговыми окнами', language=Python]{todo_with_msg.py}

\newpage
\subsection{Задание на лабораторную работу}

Для выполнения работы необходимо:
\begin{enumerate}
  \item Изучить теоретический материал
%  \item Выбрать задание в соответствии с вариатом
  \item Добавить в ранее написаную программу диалоговые окна
  \item Опубликовать код на github и предоставить ссылку в отчете
  \item Оформить отчет по лабораторной работе
\end{enumerate}

\subsection{Содержание отчета}
\begin{enumerate}
  \item Титульный Лист
  \item Цель работы
  \item Краткие теоретические сведения по теме лабораторной работы
  \item Выполненое задание
  \item Ссылка на github
  \item Краткий вывод о проделанной работе
\end{enumerate}

\begin{thebibliography}{3}
  \bibitem{python}Сузи, Р. А. Язык программирования Python : учебное пособие / Р. А. Сузи. — 3-е изд. — Москва : Интернет-Университет Информационных Технологий (ИНТУИТ), Ай Пи Ар Медиа, 2020. — 350 c. — ISBN 978-5-4497-0705-5. — Текст : электронный // Электронно-библиотечная система IPR BOOKS : [сайт]. — URL: http://www.iprbookshop.ru/97589.html
  \bibitem{tkinter} Документация стандартной библиотеки Python : tkinter — Python interface to Tcl/Tk [cайт]. --- URL: https://docs.python.org/3/library/tkinter.html
\end{thebibliography}


\end{document}
