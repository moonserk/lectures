
У нас сегодня одна из самых важных тем при этом она же одна из самых скучных. Она может выглядеть для вас несколько бюрократично, уныло,  не интересно и прочее прочее, обратите внимание на то что я буду сегодня вам рассказывать, какой то момент когда вы сами захотите сделать свой проект и сделать его не просто на коленке что-то что хоть чуть-чуть работает ,а сделать продукт который выйдет на рынок и будет использоваться людьми. Вы рано или поздно вспомните то о чем я вам говорил.

Я буду говорить про этапы проектирования какие они есть и что в них входит, что они друг под другом подразумевают. Всего этапов проектирования аж целых восемь, ну вы можете видеть их на слайде.

\begin{enumerate}
  \item Формирование требований
  \item Разработка концепций
  \item техническое задание
  \item эскизный проект
  \item Технический проект
  \item Рабочая документация
  \item Поставка / ввод в действие
  \item Сопровождение
\end{enumerate}

Кажется тут нет разработки, но потому что это все таки проектирование. Скажем так разработка она тут где-то есть внутри, но разработка не входит в проектирование. Поэтому она тут понятно не указана. Далее мы с вами будем подробно говорить про эти этапы.

Первый из них это формирование требований
на слайде написана некоторые формальные вещи которые объясняют что же такое формирование требований.

В первую очередь это обследование объекта то есть
вы должны понять собственно что и для чего вы делаете да мы не будем углубляться в какие-то формальности тем не менее если вы хотите ``'классический пример'' сервис для просмотра видео вы должны понимать как этот сервис будет выглядеть, для кого вы его делаете, кто будет им пользоваться какие есть ограничения и прочее прочее.
Либо же вы делаете messenger вы снова должны понимать для кого вы делаете что он будет выполнять. Это как-то не очень похожи на обследование, это некий анализ скорее.  Ну или что касается обследования например у вас есть какая-то заказная разработка, к вам пришел клиент и говорит сделай мне так чтобы у меня сотрудники не бегали с бумажными документами по этажам. И вы говорите мы сейчас вам сделаем электронный документооборот и все будет здорово и классно. Но вот для того чтобы понять собственно что вам надо сделать вам нужно выяснить, а какие документы у него там в компании существуют, какие процессы там существуют кто кому грубо говоря документы носит. Потому что если документы идут по некоторым процессам то есть сначала идут на подпись бухгалтеру потом директору потом еще кому то потом еще кому-то и эта линейка понятна то вам понятно как это запрограммировать, если же это происходит достаточно хаотично то как бы не понятно как это запрограммировать. Есть ситуации их там немало когда процессы не то что хаотичны, но они как минимум очень запутаны и навскидку не очень понятно как это вообще работает тогда и требуется какая-то достаточно большая работа для того чтобы понять это и каким-то образом формализовать.

Следующий этап он на самом деле практически не применим к заказной разработке так как вам пришли. Это обоснование необходимости создания вообще программного продукт ,как бы если это заказная разработка к вам пришли и сказли надо сделать в принципе вот оно и обоснование. Тут сложнее, этот процесс когда вы делаете не под заказ что-то ,а для массового рынка то есть вы предполагаете что клиенты будут, Вот он не один конкретный,а  некоторая там ниша, ну или там какой-то большая доля рынка и вы считаете что, я вот вот это сделаю и рынок будет этим пользоваться и я буду успешен --- правильный это подход. Достаточно часто во всяких там стартапах и прочего происходит так что - о классная идея побежали делать, сделали, выходит на рынок и  выясняется что это просто никому не надо, то есть это какая---то узко специализированная программа которая не решает проблем пользователей.  Которая интересна только некоторому небольшому количеству людей. \textit{Да вот феерический пример:  вот вы считаете что будет классно если я напишу программу которая будет моему соседу по комнате отправляет смску когда я захожу в какую нибудть тигру как бы зачем кому-то на рынке чтобы вашему соседу приходила эсэмэска. Это понятно что это такой гипертрофированный пример ну тут понятно как из него выйти - может быть сделать приложение которое позволит указывать номер на которую отправлять эсэмэску, тогда смысл понятен, каждый вобьет номер друга, тогда понятно кому это нужно, но кому нужно отправлять смс-ки вашему соседу. На самом деле вот этот  этап очень важен и важно как раз таки потому что если вы на него закрыли глаза и не подумали хорошенько то вы можете потратить огромное количество ресурсов, сил, времени и не дойти до того что вы хотели сделать. Более того, как помимо материальных каких-то затрат у вас могут быть и проблемы в том что и у вас и у вашей команды разработчиков наступит некоторое разочарование - типа мы делали-делали делает никому это  не нужно, ну наверное это не тот путь которым вы хотите идти. По этому не стоит игнорировать этот этап, во многом еще потому что вы можете понять что текущая идея не работает, она никому не нужна.

Следующий этап это формирование требований пользователей
Ну вот допустим снова мы делаем программу которая отсылает смску по указанному номеру телефона когда вы заходите в игру. В целом понятно что наверное это кому-то будет нужно, а вот этап формирование требований пользователей он не сколько уточняющий то есть вы понимаете да это нужно, но теперь вам нужно понять в каком формате это нужно, как оно нужно. Вы спрашиваете у пользователей ``что тебе нужно''  вот есть у меня основная идея отправляйть sms когда вы входите в игру, но пользователь скажет : в первую очередь мне нужно уметь выбирать кому отправить эсэмэску, потом  скажет что ну мне неудобно будет вбивать номер, я хочу из  контакт-листа выбирать дальше он нам скажет, а в принципе мне не нужно каждый раз выбирать из контакт-листа сделай мне возможность добавлять в избранные чтобы был какой-то шорт-лист из которого я каждый раз выбирать буду, дальше скажет я хочу  выбирать про какую игру отправлять уведомление.
Требование пользователей так называемый юзер кейс. Это кейс по которому пользователь будет пользоваться, будет использовать ваш программный продукт. При этом у вас требования могут быть не только юзер кейс, но и некоторые например по производительности то есть вам пользователь может: сказать в этой игре там матч идет 30 минут если ваша смска будет приходить через 15 минут как бы смысла в этом нет, да там мне важно чтобы это sms-ка доходил там для человека в течение там пару минут это уже требования к производительности. Допустим вывод пользователи просили вы поняли что именно им нужно какие у них есть сценарии какие у них есть требования по производительности, по качеству работы там и прочее прочее. В должны сделать по итогу этого этапа, как бы это не было скучно, вы должны написать отчетность хотя бы некоторый документик в котором вы зафиксируете все ваши умозаключения, все ваши выводы и прочее. Очень многие молодые ребята, сотрудники, ленятся это делать потому что тип отвечают что я дурак что забуду что ли, да не дурак ,но забудешь почему что пройдёт год вы сделаете этот продукт, вы его релизните например у вас пойдут негативные отзывы на какую-то функциональность, сразу возникнет вопрос а почему мы сделали так, вот документация по этапу формирования требований вам объяснит почему вы сделали так. Это некоторый артефакт к которому вы всегда можете обратится и сказать я сделал вот так потому что вот так. Если мы говорим про какую-то заказную разработку тут еще более понятно. К вам пришел дядя вася сказал сделай мне вот это, вы сделали , а при этом могут быть какие-то моменты которые на этапе общения с заказчиком, на этапе обследования вы выясняете, и вы их должны сформулировать себя внутри, потому что в будущем там на одном из этапов даже проектирования может выясниться что все не так и вы потратили кучу сил проектируя не туда, дальше у вас будет приятный разговор с заказчиком, вы придете и скажете: ты мне сказал не то, мне надо переделать. А он:я тебя не говорил. Ну как бы это вот вообще самая стандартная проблема, что если у вас идет общение с заказчиком фиксируете то о чем вы общаетесь и не потому что даже он плохой и хочет вас обмануть и не потому что вы плохие хотите обмануть, а потому что все меняется, все забывается, если у вас есть какие-то артефакты вы всегда можете к ним обратиться и это стоит того, хотя кажется это не так интересно как проектировать или писать код или писать документацию,  но тем не менее это нужно делать.

А теперь мы поговорим том какие процессы включаются в формировании требований в первую очередь это конечно же общение с клиентом, но это кажется совсем очевидно, как бы возможно понять что сделать, сформулировать какие то требования если с клиентами вы  не поговорили, обязательно нужно поговорить с пользователем.

Анализ прикладной области.
Вы можете классно знать ваш язык программирования, вы можете великолепно понимать алгоритмы, технологии, но если вы не понимаете зачем вы это делаете вы ничего не сделаете нормального

Формирование оценок требуемой производительности /качество ну тут понятно

Разработка концепции
После того как вы требование сформировали у вас наступает этап разработки концепции это тот моментскажем так когда вы должны сформулировать то что вы хотите сделать.  Должна быть концепция вы должны понимать что вы делаете она должна быть полной то есть у вас не должно быть каких-то белых мест. Вот этот модуль будет мне редактировать пользовательскую фотографию, смысле редактировать, как, обрезать или сделать квадратный, переводить в черно-белое, фильтры накладывать в instagram. Не
должно быть белых мест,тут снова включается момент что вы должны изучить прикладную область. Вы формализуете  уже то что вы наделали на предыдущем этапе. Тут же момент это относится в основном к таким к наукоемким. Это проведение необходимых мер научно-исследовательских работ то есть ну представьте сейчас к вам придут скажет а разработай софт который будет логинить пользователя по  отпечатку его локтя, конечно, легко сделаю, вот в моменте концепции когда вы еще не начали  вы должны вообще проверить можно ли по локтю то индентифицировать пользовать, локти то у всех разные или бывают одинаковые? Это научные следствие работы который не создают вам систему они лишь говорит можно это сделать и как это сделать, если они вам говорят это сделать нельзя у васв принципе накрывается весь проект.

Отчеты для НИР

Про разработку концепции
выбрать формат поставки на что это значит мы придумали классную идею да там у вас например ваш софт что он должен делать да вы говорите мой софт будет сказать это будет вот сервис человек заходит на веб страничку загружать свою фотку (десктоп, вебсервисы, мобильные)

Целевое оборудование

ТЗ это финальная четко формализация того что вы должны сделать после это и обычно ТЗ идет в комплекте с договором в договоре пишется там сроки и стоимость а в ТЗ пишется что вы должны сделать и если у вас хорошо написанный ТЗ то у вас не будет ситуаций что заказчик придет и скажет: ну вот тут подкрути, ты ему скажешь ну так дополнительнительно, ну дак ты мне сказал всю систему сделаешь, ну так делай и да вы пойдете еще допиливать, когда у вас есть жестко прописанное ТЗ вы всегда можетеапеллировать к нему аппелировать, что написано вот так, хочешь что-то еще отдельный договор, отдельные сроки, отдельные деньги.

Сточки зрения разработки почему это важно ну вот представьте вы делаете веб-сервер вы сделали под клиента веб-сервич отдаете ему уже готовый довольный собой, он звонит говорит только на главной, странице сделал розовый по розовее, вы такие что, он цвет по розовее розовый сделай, ну ладно делаете он звонит, такие ну лан, все через некоторое время, сделай еще шрифты жирные пожалуйста я слепой не не видно, и начинается вот эта вот помимо того что вы не доходите до результата вы еще тратите очень много сил на то чтобы переделать, если бы вы знали изначально что надо сделать, вот так вы бы  и делали.

1. строгое и описание системы - что мы делаем мы делаем там машину на четырех колесах рулится рулем заправляется бензином это в целом такая концепция системы
2. описание функциональности -  умеет ехать тормозить можно управлять скоростью можно поворачивать можно слушать музыку можно там не знаю печку включить а можно кондиционер, включая можно фары включить и выпишите весь функционал это очень уныло, правда, типо машина должна светить фарами ну спасибо что рассказали, да типа кто это не знает, ну не знают, там машина должна уметь ездить зимой да тут снова
3. Описание требований
4. Описание сценариев использования
5. Условия сдачи
