\documentclass[a4paper]{article}
\usepackage[T1,T2A]{fontenc}
\usepackage[utf8]{inputenc}
\usepackage[english,russian]{babel}
\usepackage{booktabs}
\usepackage{color,colortbl}
%\usepackage{amsmath}
%\usepackage{amsfonts}
%\usepackage{amssymb}
%\usepackage{makeidx}
\usepackage{listings}
\usepackage{graphicx}
\usepackage{rotating}
\definecolor{green}{RGB}{45,140,31}
\definecolor{darkishgreen}{RGB}{39,203,22}
\definecolor{LightCyan}{rgb}{0.88,1,1}
\definecolor{Gray}{gray}{0.9}
\definecolor{lightRed}{RGB}{230,170,150}
\definecolor{modRed}{RGB}{230,82,90}
\definecolor{strongRed}{RGB}{230,6,6}

\lstset{ %
  language=python,                % Язык программирования
  numbers=left,                   % С какой стороны нумеровать
  extendedchars=\true,
  %numberstyle=tinycolor{gray},     % Стиль который будет использоваться для нумерации строк
  %stepnumber=2,                   % Шаг между линиями. Если 1, то будет пронумерована каждая строка
 % numbersep=5pt,
 % backgroundcolor=color{white},      % Цвет подложки. Вы должны добавить пакет color - usepackage{color}
  showspaces=false,
  showstringspaces=false,
  showtabs=false,
  %frame=single,                    % Добавить рамку
  %rulecolor=color{black},
  tabsize=4,                       % Tab - 2 пробела
  breaklines=true,                 % Автоматический перенос строк
  breakatwhitespace=true,          % Переносить строки по словам
  title=lstname,                   % Показать название подгружаемого файла
  keywordstyle=\color{green},          % Стиль ключевых слов
  %commentstyle=color{dkgreen},       % Стиль комментариев
  %stringstyle=color{mauve}          % Стиль литералов
}

\usepackage[english,russian]{babel}

\title{\bfseriesЛабораторная работа №3.\newline

Создние приложения для работы с базой данные}
\date{}

\begin{document}

\maketitle
\newpage

Цель работы: Научиться создавать приложения с использованием базы данных.

\section{Теоретические основы}

Разработаная ранее программа 'Список дел' имеет большой недостаток --- при закрытии программы все данные теряются. Что бы сохранять данные можно использовать текстовый файл в который будут записыватся данные при выключении, а при включениии данные будут подтягиватся из файла. Но это не оптимальный вариант решения проблемы так как это не позволит масштабировать проект в будущем и вносит массу ограничений для манипулирования данными. Для хранения данных в разработке ПО чаще всего пользуются \textbf{базами данных}.

\textbf{База данных} --- это набор перманентных данных организованный в соответствии с определенной схемой, манипулирование которыми выполняются в соответствии с правилами средств моделирования данных. Так же существует \textbf{система управления базами данных}. СУБД --- это комплекс программ, позволяющих создать и манипулировать данными (вставлять, обновлять, удалять и выбирать). Наиболее распространными на сегодняшний день являются реляционные базы данных такие как Oracle, MySQL, PostgreSQL, MS SQL, SQLite.

Для легких проектов хорошо подйдет база данных \textbf{SQLite} так как она предоставляет легкую дисковую базу данных, не требующую отдельного серверного процесса и позволяющую получить доступ к базе данных с использованием нестандартного варианта языка запросов SQL. Некоторые приложения могут использовать SQLite для внутреннего хранения данных. Также можно создать прототип приложения с помощью SQLite, а затем перенести код в более крупную базу данных, такую как PostgreSQL или Oracle.

Что бы работать с реляционными базами данных необходимо знать язык запросов (SQL).
Язык SQL подразделяется на несколько частей, рассмотрим 2 наиболее важные его части:
\begin{enumerate}
  \item DDL – Data Definition Language (язык описания данных), который содержит следующие конструкции:
    \begin{itemize}
      \item CREATE --- для создания базы данных и ее объектов, таких как (таблица, индекс, представления, процедура хранения, функция и триггеры)
      \item ALTER --- изменяет структуру существующей базы данных
      \item DROP --- удалить объекты из базы данных
      \item TRUNCATE --- удалить все записи из таблицы, включая все места, выделенные для записей, удалены
      \item КОММЕНТАРИЙ --- добавить комментарии в словарь данных
      \item RENAME --- переименовать объект
    \end{itemize}
  \item DML – Data Manipulation Language (язык манипулирования данными), который содержит следующие конструкции:
    \begin{itemize}
        \item SELECT --- выборка данных
        \item INSERT --- вставка новых данных
        \item UPDATE --- обновление данных
        \item DELETE --- удаление данных
        \item MERGE --- слияние данных
    \end{itemize}
\end{enumerate}

Например, в листинге~\ref{create} создается таблица \textbf{tasks} (при условии что такой еще не существует) с полями \textbf{title} и \textbf{text}:

\begin{lstlisting}[label=create,caption=Создание таблицы]
  create table if not exists
  tasks (title text)
\end{lstlisting}

Для вставки данных в таблицу используется конструкция \textbf{INSERT}, например, Листинге~\ref{insert} в таблицу tasks вставляется текст 'Выпить чаю'

\begin{lstlisting}[label=insert,caption=Добавление новой записи]
  insert into tasks
  values ('Выпить чаю')
\end{lstlisting}

Для удаления данных из таблицы используется конструкция \textbf{DELETE} (см. Листинг~\ref{delete})

\begin{lstlisting}[label=delete,caption=Создание таблицы]
  delete
  from tasks where title = 'Выпить чаю'
\end{lstlisting}

Что бы запросить все данные из таблицы необходимо воспользоватся конструкцией \textbf{SELECT}, например в Листинге~\ref{select} производится выборка всех записей поля title из таблицы tasks.

\begin{lstlisting}[label=select,caption=Выборка данных]
  select title
  from tasks
\end{lstlisting}

Что бы подключиться и использовать базу данных необходимо:
\begin{enumerate}
  \item импортровать библиотеку sqlite3
  \item создать подлючение к базе данных: \textit{connection = sqlite3.connect(``todo.db'')}
  \item создть обект курсора для управления базой данных \textit{cursor = connection.cursor()}
\end{enumerate}

Метод объекта \textsf{cursor.execute()} принимает аргумент строки с SQL --- запросом, и возвращает результат.

\lstinputlisting[caption='Список дел с базой данных', language=Python]{todo_with_db.py}

Теперь после выключения программы данные сохраняются в базе данных.

\newpage
\section{Задание на лабораторную работу}

Для выполнения работы необходимо:
\begin{enumerate}
  \item Изучить теоретический материал
%  \item Выбрать задание в соответствии с вариатом
  \item Добавить в ранее написаную программу работу с базой данных
    \begin{itemize}
      \item Можно использовать любую базу данных
    \end{itemize}
  \item Опубликовать код на github и предоставить ссылку в отчете
  \item Оформить отчет по лабораторной работе
\end{enumerate}

\section{Содержание отчета}
\begin{enumerate}
  \item Титульный Лист
  \item Цель работы
  \item Краткие теоретические сведения по теме лабораторной работы
  \item Выполненое задание
  \item Ссылка на github
  \item Краткий вывод о проделанной работе
\end{enumerate}

\begin{thebibliography}{3}
  \bibitem{python}Сузи, Р. А. Язык программирования Python : учебное пособие / Р. А. Сузи. — 3-е изд. — Москва : Интернет-Университет Информационных Технологий (ИНТУИТ), Ай Пи Ар Медиа, 2020. — 350 c. — ISBN 978-5-4497-0705-5. — Текст : электронный // Электронно-библиотечная система IPR BOOKS : [сайт]. — URL: http://www.iprbookshop.ru/97589.html
  \bibitem{db}Кара-Ушанов, В. Ю. SQL - язык реляционных баз данных : учебное пособие / В. Ю. Кара-Ушанов. — Екатеринбург : Уральский федеральный университет, ЭБС АСВ, 2016. — 156 c. — ISBN 978-5-7996-1622-9. — Текст : электронный // Электронно-библиотечная система IPR BOOKS : [сайт]. — URL: http://www.iprbookshop.ru/68419.html 
  \bibitem{tkinter} Документация стандартной библиотеки Python : sqlite3 — Python sqlite3 [cайт]. --- URL: https://docs.python.org/3/library/sqlite3.html
  \bibitem{sqlite} Документация sqlite [сайт]. --- URL: https://sqlite.org/docs.html
\end{thebibliography}


\end{document}
