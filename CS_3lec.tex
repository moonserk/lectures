\documentclass[a4paper]{article}
\usepackage[T1,T2A]{fontenc}
\usepackage[utf8]{inputenc}
\usepackage[english,russian]{babel}
\usepackage{booktabs}
\usepackage{color,colortbl}
%\usepackage{amsmath}
%\usepackage{amsfonts}
%\usepackage{amssymb}
%\usepackage{makeidx}

\definecolor{darkishgreen}{RGB}{39,203,22}
\definecolor{LightCyan}{rgb}{0.88,1,1}
\definecolor{Gray}{gray}{0.9}
\definecolor{lightRed}{RGB}{230,170,150}
\definecolor{modRed}{RGB}{230,82,90}
\definecolor{strongRed}{RGB}{230,6,6}

\usepackage[english,russian]{babel}

\begin{document}

\section{Информационные ресурсы. Информационные системы и технологии}

\subsection{Информационные ресурсы}

Информационные ресурсы – это идеи человечества и указания по их реализации, накопленные в форме, позволяющей их воспроизводство. Это книги, статьи, патенты, диссертации, научно-исследовательская и опытно-конструкторская документация, технические переводы, данные о передовом производственном опыте и др. Информационные ресурсы (в отличие от всех других видов ресурсов — трудовых, энергетических, минеральных и т.д.) тем быстрее растут, чем больше их расходуют.

Система – любой объект, который одновременно рассматривается и как единое целое, и как объединенная в интересах достижения поставленной цели совокупность разнородных элементов.
Основные понятия:
\begin{enumerate}
\item Элемент системы – часть системы, имеющая определенное функциональное назначение. Сложные элементы, в свою очередь состоящие из более простых взаимосвязанных элементов, часто называют подсистемами.
\item  Организация системы – внутренняя упорядоченность, согласованность взаимодействия элементов системы, проявляющаяся, в частности, в ограничении разнообразия состояний элементов системы.
\item Структура системы – состав, порядок и принципы взаимодействия элементов системы, определяющая ее основные свойства. (например – иерархическая).
\item Архитектура системы – совокупность свойств системы, существенных для пользователя.
\item Целостность системы – принципиальная несводимость свойств системы к сумме свойств отдельных ее элементов и в то же время зависимость свойств каждого элемента от его места и функции внутри системы.
\end{enumerate}

\subsection{Процессы в информационной системе}
Информационная система – взаимосвязанная совокупность средств, методов и персонала, используемых для хранения, обработки и выдачи информации в интересах достижения поставленной цели.
  ИС определяется следующими свойствами:
  \begin{itemize}
    \item Любая ИС может быть подвергнута анализу, построена и управляема на основе общих принципов построения систем;
    \item ИС является динамичной и развивающейся;
    \item При построении ИС необходимо использовать системный подход;
    \item Выходной продукцией ИС является информация, на основе которой принимаются решения;
    \item ИС следует понимать как человеко-компьютерную систему обработки информации.
  \end{itemize}

  Структуру ИС составляет совокупность отдельных ее частей, называемых подсистемами. Подсистема – это часть системы, выделенная по какому-либо признаку.Подсистемы ИС называют обеспечивающими. Среди них выделяют:
  \begin{enumerate}
    \item Информационное обеспечение (informational support) – система классификации и кодирования информации, технологическая схема обработки данных, нормативно-справочная информация, система документооборота.
    \item Организационное обеспечение (organizational support) – совокупность мер и мероприятий, регламентирующих функционирование системы управления, взаимодействие работников с техническими средствами и между собой, описание системы, инструкции и регламенты обслуживающему персоналу.
    \item Техническое обеспечение (hardware) – комплекс используемых в системе управления технических средств, включающий ЭВМ и средства связи, а также документация на эти средства и технологические  процессы.
    \item Математическое обеспечение (mathematical support) – совокупность методов, правил, математических моделей и алгоритмов решения задач.
    \item Лингвистическое обеспечение (linguistic support) – совокупность терминов и искусственных языков, правил формализации естественного языка.
    \item Программное обеспечение (software) – совокупность программ систем обработки данных и документов, необходимых для эксплуатации этих программ. (Системное и прикладное ПО).
    \item Правовое обеспечение (legal support) – совокупность правовых норм, определяющих создание, юридический статус и функционирование системы, регламентирующих порядок получения, преобразования и использования информации.
  \end{enumerate}

  Информационная технология (ИТ)
  \begin{itemize}
      \item система процедур преобразования информации с целью формирования, организации, обработки, распространения и использования информации;
      \item процесс, использующий совокупность средств и методов сбора, обработки и передачи данных (первичной информации) для получения информации нового качества о состоянии объекта, процесса или явления.
\end{itemize}
Этапы ИТ (виды инструментария):
\begin{enumerate}
  \item Ручная  (до 2-й половины 19 в.) --- Перо, чернила, бумага. Почта, курьеры.
  \item Механическая (с кон. 19 в.). --- Пишущая машинка, телефон, диктофон. Почта с более совершенными средствами доставки.
  \item Электрическая (40 --- 60е гг. 20в.) --- Большие ЭВМ, электр. Пиш. Машинки, портативные диктофоны, ксероксы.
  \item  Электронная (с нач. 70-х гг. 20в.). --- АСУ и информационно-поисковые системы (ИПС) на базе больших ЭВМ.
  \item Новая (с сер. 80-х гг. 20в.). --- Персональные ЭВМ, персонализация АСУ, системы искусственного интеллекта для поддержки принятия решений.
\end{enumerate}

Основу современных ИТ составляют:
\begin{itemize}
  \item Компьютерная обработка информации по заданным алгоритмам;
  \item Хранение больших объемов информации на машинных носителях;
  \item Передача информации на любое расстояние в ограниченное время.
\end{itemize}

\end{document}
